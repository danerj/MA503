\documentclass[a4paper]{article}

%% Language and font encodings
\usepackage[english]{babel}
\usepackage[utf8x]{inputenc}
\usepackage[T1]{fontenc}

%% Sets page size and margins
\usepackage[a4paper,top=3cm,bottom=2cm,left=3cm,right=3cm,marginparwidth=1.75cm]{geometry}

%% Useful packages
\usepackage{amsmath}
\usepackage{graphicx}
\usepackage[colorinlistoftodos]{todonotes}
\usepackage[colorlinks=true, allcolors=blue]{hyperref}
\usepackage{float}
\usepackage{enumerate}
\usepackage{subfig}
\setlength\parindent{0pt}
\usepackage{amssymb}



\makeatletter
\def\moverlay{\mathpalette\mov@rlay}
\def\mov@rlay#1#2{\leavevmode\vtop{%
   \baselineskip\z@skip \lineskiplimit-\maxdimen
   \ialign{\hfil$\m@th#1##$\hfil\cr#2\crcr}}}
\newcommand{\charfusion}[3][\mathord]{
    #1{\ifx#1\mathop\vphantom{#2}\fi
        \mathpalette\mov@rlay{#2\cr#3}
      }
    \ifx#1\mathop\expandafter\displaylimits\fi}
\makeatother

\newcommand{\cupdot}{\charfusion[\mathbin]{\cup}{\cdot}}
\newcommand{\bigcupdot}{\charfusion[\mathop]{\bigcup}{\cdot}}

\title{MA 503 : Homework 17}
\author{Dane Johnson}

\begin{document}
\maketitle

{\bf Lemma Eggcorn} If $A$ and $B$ are sets of real numbers bounded below, then $\inf(A+B)= \inf(A) + \inf(B)$. \\

Proof: Let $a+b \in A+B$ with $a \in A$ and $b \in B$. Since $\inf(A) \leq a$ and $\inf(B) \leq b$, $\inf(A) + \inf(B) \leq a+b$. So $\inf(A) + \inf(B)$ is a lower bound of $A+B$ and $\inf(A) + \inf(B) \leq \inf(A+B)$. For each $\epsilon > 0$ there is exist $a \in A$ and $b \in B$ such that $a< \inf(A) + \epsilon / 2$ and $b < \inf(B) + \epsilon /2$ so that $\inf(A+B) \leq a+b < \inf(A) + \inf(B) + \epsilon$. Then since $\inf(A+B) < \inf(A) +\inf(B) + \epsilon$ for every $\epsilon > 0$, $\inf(A+B) \leq \inf(A) + \inf(B)$. \\


{\bf Problem 1} Show that $||f+g||_\infty \leq ||f||_\infty + ||g||_\infty$.\\


$$A := \{P : m(\{t: |f(t) + g(t)| > P\}) = 0\} = \{P : |f+g| \leq P \text{ a.e.} \}$$
$$B := \{M : m(\{t: |f(t)| > M\}) = 0\} = \{M : |f| \leq M \text{ a.e.} \}$$
$$C := \{N : m(\{t: |g(t)| > N\}) = 0\} = \{N : |g| \leq N \text{ a.e.} \}$$
$$B+C := \{M+N : m(\{t: |f(t)| > M\}) = 0, m(\{t: |g(t)| > N\}) = 0\} = \{M+N : |f| \leq M \text{ a.e.}, |g| \leq N \text{ a.e.} \}\;.$$

Let $M+N \in B+C$ such that $M \in B$ and $N \in C$. Then $|f| \leq M$ a.e. and $|g| \leq N$ a.e. which means that $|f| + |g| \leq M+N$ a.e. But then $|f+g| \leq |f| + |g| \leq M+N$ a.e. so that $M+N \in A$. Therefore $B+C \subset A$ and $\inf A \leq \inf (B+C)$. Since $B$ and $C$ are sets of real numbers bounded each bounded below by 0 (because $|f|, |g| \geq 0$), $\inf (B+C) = \inf B + \inf C$ by Lemma Eggcorn. That is,

\begin{align*}
||f + g||_\infty &= \inf \{P : m(\{t: |f(t) + g(t)| > P\}) = 0\} \\
&= \inf A\\
& \leq \inf B + \inf C\\
& = \inf \{M : m(\{t: |f(t)| > M\}) = 0\} + \inf \{N : m(\{t: |g(t)| > N\}) = 0\}\\
& = ||f||_\infty + ||g||_\infty \;.
\end{align*}

We can also stick with the provided definition of essential supremum and follow a similar but messier argument. If $M \in B$ and $N \in C$, then $m(\{t : |f(t)| > M\}) = 0$ and $m(\{t : |f(t)| > N \}) = 0$.
\begin{align*}
& m(\{t : |f(t) + g(t)| > M + N\}) \\ &\leq m(\{t : |f(t)| + |g(t)| > M + N \}) \quad \text{(from the triangle inequality)}\\
&= m[(\{t : |f(t)| > M\} \cap \{t: |g(t)| > N\})\cup \\ &(\{t : |g(t)| \leq N \} \cap \{t : |f(t)| > M + N - |g(t)|\}) \cup (\{t : |f(t)| \leq M \} \cap \{t : |g(t)| > M + N - |f(t)|\})]\\
&\leq m(\{t : |f(t)| > M\} \cap \{t: |g(t)| > N\})
+ m(\{t : |g(t)| \leq N \} \cap \{t : |f(t)| > M + N - |g(t)|\}) \\ &+ m(\{t : |f(t)| \leq M \} \cap \{t : |g(t)| > M + N - |f(t)|\}) \\
& \leq m(\{t : |f(t)| > M\}) + m(\{t : |f(t)| > M\}) + m(\{t : |g(t)| > N\}) \\
&= 0 \;.
\end{align*}

This shows that if $M+N \in B+C$ with $M \in B$ and $N \in C$, then $M+N \in A$. Therefore, $B+C \subset A$ and so $\inf(A) \leq \inf(B+C) = \inf(B) + \inf(C)$. That is, $||f+g||_\infty \leq ||f||_\infty + ||g||_\infty$. \\

{\bf Problem 2} Let $f$ be a bounded measurable function on $[0,1]$. Prove that $\lim_{p\rightarrow \infty} ||f||_p = ||f||_\infty$. \\

We have $|f| \leq ||f||_\infty$ a.e. and so

\begin{align*}
||f||_p &= \left(\int_0^1 |f|^p\right)^{1/p}\\
&\leq \left(\int_0^1 ||f||_\infty^p \right)^{1/p}\\
&= \left(||f||_\infty^p m([0,1])\right)^{1/p} \\
&= ||f||_\infty\\
\implies \lim_{p\rightarrow \infty} ||f||_p &\leq \||f||_\infty \;. 
\end{align*}

For each $p \in \mathbb{N}$, the set $B_p = \{x : |f(x)| > ||f||_\infty - 1/p\}$ has positive measure since if $m(B_p) = 0$, we have $||f||_\infty - 1/p \in \{M : m(\{x : |f(x)| > M \}) = 0 \}$ and $||f||_\infty - 1/p < ||f||_\infty = \inf \{M : m(\{x : |f(x)| > M \}) = 0 \}$, which is a contradiction. 

\begin{align*}
\left(\int_{B_p} |f|^p \right)^{1/p}& \geq \left(\int_{B_p} |\;||f||_\infty - 1/p \;|^p\right)^{1/p}\\
&=| \; ||f||_\infty - 1/p \;| \left(m(B_p)\right)^{1/p}\\
\end{align*}

As $p \rightarrow \infty$, $||f||_\infty - 1/p \rightarrow ||f||_\infty$, so $B_p \rightarrow [0,1]$. Then,

$$\lim_{p \rightarrow \infty} ||f||_p \geq \lim_{p \rightarrow \infty} | \; ||f||_\infty - 1/p \; | \left(m(B_p)\right)^{1/p} = ||f||_\infty m([0,1]) = ||f||_\infty \;.$$


{\bf Problem 3} Prove that $||f + g||_1 \leq ||f||_1 + ||g||_1$. \\

Suppose $f,g \in L^1([0,1])$. For each $x \in [0,1]$, $|f+g| \leq |f| + |g|$, so

$$||f+g||_1 = \int_0^1 |f+g| \leq \int_0^1 (|f| + |g|)= \int_0^1 |f| + \int_0^1 |g|  = ||f||_1 + ||g||_1 \;.$$

{\bf Problem 4} Show that if $f \in L^1$ and $g \in L^\infty$,

$$\int |fg| \leq ||f||_1 \cdot ||g||_\infty \;.$$

We have $|g| \leq ||g||_\infty$ almost everywhere so $|fg| = |f||g| \leq |f| ||g||_\infty$ almost everywhere. We have defined $L^p$ spaces in this section on the interval $[0,1]$, so

$$\int |fg| = \int_0^1 |fg| = \int_0^1 |f||g| \leq \int_0^1 |f| ||g||_\infty = ||g||_\infty \int_0^1 |f| = ||g||_\infty \int |f| = ||g||_\infty ||f||_1 \;.$$


\end{document}