\documentclass[a4paper]{article}

%% Language and font encodings
\usepackage[english]{babel}
\usepackage[utf8x]{inputenc}
\usepackage[T1]{fontenc}

%% Sets page size and margins
\usepackage[a4paper,top=3cm,bottom=2cm,left=3cm,right=3cm,marginparwidth=1.75cm]{geometry}

%% Useful packages
\usepackage{amsmath}
\usepackage{graphicx}
\usepackage[colorinlistoftodos]{todonotes}
\usepackage[colorlinks=true, allcolors=blue]{hyperref}
\usepackage{float}
\usepackage{enumerate}
\usepackage{subfig}
\setlength\parindent{0pt}
\usepackage{amssymb}



\makeatletter
\def\moverlay{\mathpalette\mov@rlay}
\def\mov@rlay#1#2{\leavevmode\vtop{%
   \baselineskip\z@skip \lineskiplimit-\maxdimen
   \ialign{\hfil$\m@th#1##$\hfil\cr#2\crcr}}}
\newcommand{\charfusion}[3][\mathord]{
    #1{\ifx#1\mathop\vphantom{#2}\fi
        \mathpalette\mov@rlay{#2\cr#3}
      }
    \ifx#1\mathop\expandafter\displaylimits\fi}
\makeatother

\newcommand{\cupdot}{\charfusion[\mathbin]{\cup}{\cdot}}
\newcommand{\bigcupdot}{\charfusion[\mathop]{\bigcup}{\cdot}}

\title{MA 503 : Homework 5}
\author{Dane Johnson}

\begin{document}
\maketitle


{\bf Definition} Let $X$ be a set and $\mathfrak{M}$ a collection of subsets of $X$, that is $\mathfrak{M} \subset \mathcal{P}(X)$. Then $\mathfrak{M}$ is a {\bf $\sigma$ - algebra} if

($\Sigma_1$) $X \in \mathfrak{M}$;\\
($\Sigma_2$) $A \in \mathfrak{M} \implies A^c \in \mathfrak{M}$;\\
($\Sigma_3$) For any sequence of sets $(A_n)\subset \mathfrak{M}$, $\cup_{n=1}^\infty A_n \in \mathfrak{M}$.\\

Considering ($\Sigma_2$), ($\Sigma_1$) could instead require $\emptyset \in \mathfrak{M}$ or both $\emptyset, X \in \mathfrak{M}$. \\

{\bf Definition / Proposition} Let $\mathcal{F} \subset \mathcal{P}(X)$. There is a smallest $\sigma$-algebra $\mathcal{C}$ such that $\mathcal{F} \subset \mathcal{C}$. We call $\mathcal{C}$ the {\bf $\sigma$-algebra generated by the collection of sets $\mathcal{F}$}. \\

Proof: Let $\mathcal{C} = \bigcap\{\mathcal{A} : \mathcal{A} \text{ is a } \sigma \text{ algebra } \text{ and } \mathcal{F}\subset \mathcal{A}\}$. Note that the $\mathcal{A}$ are collections of sets so that an intersection would yield the collection of sets common to all such $\mathcal{A}$.  Claim that (i) $\mathcal{C}$ is a $\sigma$-algebra, (ii) $\mathcal{F}\subset \mathcal{C}$ and (iii) $\mathcal{C}$ is the smallest $\sigma$-algebra containing $\mathcal{F}$.\\

Since $X \in \mathcal{A}$ for all $\mathcal{A}$ used in the intersection, $X \in \mathcal{C}$. Let $S \in \mathcal{C}$ Then $S \in \mathcal{A}$ for all $\mathcal{A}$. Since all $\mathcal{A}$ are $\sigma$-algebras $S^c \in A$ for all $\mathcal{A}$. So $S^c \in \mathcal{C}$. If $(S_n)$ is a sequence of sets in $\mathcal{C}$, then $(S_n)\subset \mathcal{A}$ for all $\mathcal{A}$ and so $\bigcup S_n \in \mathcal{A}$ for all $\mathcal{A}$. Thus $\bigcup S_n \in \mathcal{C}$.\\

Let $S \in \mathcal{F}$ ($S$ a set in the set of sets $\mathcal{F}$). Then since $\mathcal{F}\subset \mathcal{A}$ for every $\mathcal{A}$ used in the intersection, $S \in \mathcal{A}$ for all $\mathcal{A}$ and so $S \in \mathcal{C}$. Thus $\mathcal{F}\subset \mathcal{C}$. \\

We have seen that $\mathcal{C}$ is a $\sigma$-algebra that contains $\mathcal{F}$. Suppose $\mathcal{D}$ is any $\sigma$-algebra  and $\mathcal{F}\subset \mathcal{D}$. Then $\mathcal{D} \in \{\mathcal{A} : \mathcal{A} \text{ is a } \sigma \text{ algebra } \text{ and } \mathcal{F}\subset \mathcal{A}\}$ by definition of this set and therefore $\mathcal{C} = \bigcap \{\mathcal{A} : \mathcal{A} \text{ is a } \sigma \text{ algebra } \text{ and } \mathcal{F}\subset \mathcal{A}\} \subset \mathcal{D}$. That is $\mathcal{C}$ is an admissible $\sigma$-algebra and is a subset of any other admissible $\sigma$-algebra. In this sense $\mathcal{C}$ is the smallest $\sigma$-algebra containing $\mathcal{F}$.\\


{\bf Unspecified Homework 5 Problem} For $\mathcal{F} = \{S \subset \mathbb{R} : S \text{ is finite }\}$ find the smallest $\sigma$-algebra containing $\mathcal{F}$ as a subcollection. To say that a $\sigma$-algebra contains $\mathcal{F}$ as a subcollection means that the $\sigma$-algebra contains all of the sets in $\mathcal{F}$, which is itself a set of sets but not necessarily a $\sigma$-algebra. \\

Let $\sigma(\mathcal{F})$ denote the $\sigma$-algebra generated by $\mathcal{F}$. By the Definition / Proposition above $\sigma(\mathcal{F})$ is the smallest $\sigma$-algebra containing $\mathcal{F}$ as a subcollection. It will be shown that $\sigma(\mathcal{F}) = \{A \subset \mathbb{R} : A \text{ is countable or } A^c \text{ is countable }\}$. Here we take the definition of countable to mean finite or countably infinite. Sometimes the definition of countable is taken to mean specifically the case of countably infinite but that is not the definition to be considered here.\\

Proof: We will prove that $\sigma(\mathcal{F})$ as defined is indeed a $\sigma$-algebra with $\mathcal{F} \subset\sigma(F)$ and $\sigma(\mathcal{F})$ is the smallest $\sigma$-algebra containing $\mathcal{F}$ as a subcollection. Since these claims have not been established yet, let $\mathcal{C} = \{A \subset \mathbb{R} : A \text{ is countable or } A^c \text{ is countable }\}$.\\

First the claim that $\mathcal{C}$ is a $\sigma$-algebra will be verified. Since $\mathbb{R}^c = \emptyset$ and the empty set is countable as $|\emptyset| = 0$, $\mathbb{R} \in \mathcal{C}$. If $C \in \mathcal{C}$, then either $C$ is countable or $C^c$ is countable. If $C$ is countable, then since $(C^c)^c = C$, $C^c \in \mathcal{C}$ (whether or not $C^c$ is countable, the fact that $(C^c)^c$ is countable is enough to show that $C^c \in \mathcal{C}$). If $C$ is not countable, the assumption that $C\in \mathcal{C}$ means that $C^c$ must be countable. Then $C^c \in \mathcal{C}$. Suppose $(C_n)$ is a sequence of sets with $C_n \in \mathcal{C}$ for each $n \in \mathbb{N}$. Either all $C_n$ in the sequence $(C_n)$ are countable or there is at least one uncountable set $C_k$ in the sequence. If all $C_n$ are countable, then the countable union of countable sets $\bigcup_{n \in \mathbb{N}} C_n$ is also countable so $\bigcup_{n \in \mathbb{N}} C_n \in \mathcal{C}$. If it is not the case that $C_n$ is countable for all $n$, then there is at least one set uncountable set $C_{k}$ in the sequence. Since $C_k \in \mathcal{C}$ and $C_k$ is not countable, $C_k^c$ must be countable. Then $\bigcup_{n\in \mathbb{N}} C_n \supset C_k$ is not countable but since $\left(\bigcup_{n \in \mathbb{N}} C_n\right)^c = \bigcap_{n\in \mathbb{N}} C_n^c \subset C_k^c$ and $C_k^c$ is countable, $\left(\bigcup_{n \in \mathbb{N}} C_n\right)^c$ is countable. This means that $\bigcup_{n \in \mathbb{N}} C_n \in \mathcal{C}$. Conclude that $\mathcal{C}$ is a $\sigma$-algebra. \\

Next the claim that $\mathcal{F} \subset \mathcal{C}$ will be verified. Let $F \in \mathcal{F}$. Then $F$ is a finite set. A finite set is countable under our definition of countability. Therefore $F \in \mathcal{C}$. Since $F \in \mathcal{F}$ was arbitrary, $\mathcal{F}\subset \mathcal{C}$.  \\

Finally the claim that $\mathcal{C}$ is the smallest $\sigma$-algebra containing $\mathcal{F}$ as a subcollection will be verified. Suppose that $\mathcal{C}'$ is a $\sigma$-algebra containing $F$ as a subcollection. We will show that $\mathcal{C}\subset \mathcal{C}'$. Suppose for contradiction that $\mathcal{C} \not\subset \mathcal{C}'$ and that $C \in \mathcal{C}$ but $C \not\in \mathcal{C}'$. Either $C$ is countable or $C^c$ is countable. If $C$ is countable, we may order the elements of $C$ as $C = \{c_1,...,c_n\}$ if $C$ is finite or $C = \{c_1,c_2,c_3,...\}$ if $C$ is countably infinite. In the case that $C$ is finite then $C \in \mathcal{F}$ but $C \not\in \mathcal{C}'$, which contradicts the assumption that $\mathcal{C}'$ contains $\mathcal{F}$ as a subcollection. If $C$ is countably infinite, note that the singletons $\{c_i\}, i \in \mathbb{N}$ are all finite (and so elements of $\mathcal{F}$) and since $\mathcal{C}'$ is a $\sigma$-algebra containing every set in $\mathcal{F}$, $C = \bigcup_{i\in \mathbb{N}} c_i \in \mathcal{C}'$, which contradicts the assumption that $C \not\in \mathcal{C}'$. If $C$ is not countable then $C^c$ is countable. By similar reasoning, it follows that $C^c \in \mathcal{C}'$. But since $\mathcal{C}'$ is a $\sigma$-algebra, this again gives the contradiction that $C \in \mathcal{C}'$. The assumption that $\mathcal{C} \not\subset \mathcal{C}'$ leads in any case to a contradiction, so $\mathcal{C} \subset \mathcal{C}'$ for any $\sigma$-algebra $\mathcal{C}'$ that contains $\mathcal{F}$ as a subcollection. In this sense $\mathcal{C}$ is the minimal (or smallest) $\sigma$-algebra that contains $\mathcal{F}$ as a subcollection.\\

Since $\mathcal{C} =\{A \subset \mathbb{R} : A \text{ is countable or } A^c \text{ is countable }\}$ is a $\sigma$-algebra containing $\mathcal{F}$ as a subcollection and $\mathcal{C} \subset \mathcal{C}'$ for any $\sigma$-algebra $\mathcal{C}'$ that contains $\mathcal{F}$ as a subcollection, $\mathcal{C}$ is the smallest $\sigma$-algebra containing $\mathcal{F}$ as a subcollection and we may write $\mathcal{C} = \sigma(\mathcal{F})$ to specify that this is the $\sigma$-algebra generated by $\mathcal{F}$. \\



{\bf Definition} Let $\mathfrak{M}$ be a $\sigma$ algebra  of sets (or real numbers). A function $m$ is called a {\bf countably additive measure} if\\

($M_1$) $m : \mathfrak{M}\rightarrow [0,+\infty]$\\
($M_2$) For any  sequence of disjoint sets $(E_n) \subset \mathfrak{M}$, $m(\bigcupdot E_n) = \sum m(E_n)$.\\

In problems 1 - 3 let $m$ be a countably additive measure  defined for all sets in a $\sigma$ algebra $\mathfrak{M}$. Whether this omission is an error or intentional, there is no evidence that $m(\emptyset) := 0$, so it is necessary to prove some lemmas to handle this inconvenience. Problem 3 essentially handles this but I was unsure how else to complete Problem 1 without using the measure of the empty set in some way.  \\

{\bf Lemma 1} $m(\emptyset) = 0$ or $m(\emptyset) = \infty$.\\

Proof: Suppose $ m(\emptyset) = \alpha \in (0,\infty)$. Then since $\emptyset = \bigcupdot_{i = 1}^\infty \emptyset$, we obtain $\alpha = m(\emptyset) = m(\bigcupdot_{i = 1}^\infty \emptyset) = \sum_{i=1}^\infty m(\emptyset) = \sum_{i=1}^\infty \alpha = \infty$. This contradicts the assumption that $\alpha < \infty$. Therefore $m(\emptyset) = 0$ or $m(\emptyset) = \infty$ \\

{\bf Lemma 2} If $m(\emptyset) = \infty$, $m(A) = \infty$ for each $A$ in $\mathfrak{M}$. \\

The sequence $(E_n)$ with $E_1 = A$, $E_n = \emptyset$ for $n\geq 2$ is a disjoint sequence. By $M_2$,
$$m(A) = m\left(\bigcupdot E_n\right) = \sum E_n = m(A) + \sum_{n\geq 2} E_n  \geq 0 + \sum_{n\geq 2} E_n = \infty \implies m(A) = \infty \;.$$


{\bf Problem 1} If $A$ and $B$ are two sets in $\mathfrak{M}$ with $A \subset B$, then $m(A) \leq m(B)$. This property is called {\bf monotonicity}.\\


Proof: If $m(\emptyset) = \infty$ the inequality holds by Lemma 2. Otherwise assume this is not the case so that $m(\emptyset) = 0$ by Lemma 2. Since $A\subset B$, $A\cup B = B$. The sets $A$ and $B\backslash A$ are disjoint since $A \cap (B \backslash A) = A \cap (B \cap A^c) = (A\cap A^c) \cap B = \emptyset \cap B = \emptyset$ while $A \cupdot (B\backslash A) = A\cupdot(B \cap A^c) = (A\cup B) \cap (A \cupdot A) = (A\cup B) \cap \mathbb{R} = A\cup B = B$. Also, $\emptyset \cap A = \emptyset$, $\emptyset \cap  (B\backslash A) = \emptyset$, and $\emptyset\cap \emptyset = \emptyset$. Define the sequence $(E_n)$ by $E_1 = A$, $E_2 = B\backslash A$ and $E_n = \emptyset$ for $n \geq 3$. Then $(E_n)$ is a sequence of disjoint sets and by $M_2$,
$$0 \leq m(B) = m(A\cupdot (B \backslash A) = m(\bigcupdot E_n) = \sum m(E_n) = m(A) + m(B\backslash A) + 0 + 0 + ... = m(A) + m(B\backslash A) \;.$$

This shows that $m(A) + m(B\backslash A) = m(B)$. Since $m(B\backslash A) \geq 0$, $m(B) \geq m(B) - m(B\backslash A) = m(A)$. Therefore $m(A) \leq m(B)$. \\


{\bf Problem 2} Let $(E_n)$ be a sequence of sets in $\mathfrak{M}$. Then $m(\bigcup E_n) \leq \sum m(E_n)$. This property is called {\bf countable subadditivity}. \\

Proof: If $m(\emptyset) = \infty$ the inequality holds immediately. Assume this is not the case so that $m(\emptyset) = 0$. Define $E_0 : = \emptyset$ for notational consistency and the sequence $(F_n)$ by
\begin{align*}
F_1 &= E_1 \backslash E_0 \\
F_2 &= E_2 \backslash F_1 = E_2 \backslash (E_0 \cup E_1)\\
F_3 &= E_3 \backslash (E_0 \cup E_1 \cup E_2) \\
F_4 &= E_4 \backslash (E_0 \cup E_1 \cup E_2 \cup E_3) \\
&... \\
F_n &= E_n \backslash (\bigcup_{i=0}^{n-1} E_i)
\end{align*}

Suppose $m\neq n$ and without loss of generality that $n < m$.

\begin{align*}
F_m \cap F_n &= \left(E_m \cap (\bigcup_{i=0}^{m-1} E_i)^c\right) \cap \left(E_n \cap (\bigcup_{i=0}^{n-1} E_i)^c\right)\\
&=\left(E_m \cap (\bigcap_{i=0}^{m-1} E_i^c)\right) \cap \left(E_n \cap (\bigcap_{i=0}^{n-1} E_i^c)\right)\\
&=\left(E_m \cap (E_n^c \cap \bigcap_{i=0, i \neq n}^{m-1} E_i^c)\right) \cap \left(E_n \cap (\bigcap_{i=0}^{n-1} E_i^c)\right)\\
&=\left(E_m \cap (\bigcap_{i=0, i \neq n}^{m-1} E_i^c)\right) \cap (E_n^c \cap E_n) \cap \left(\cap (\bigcap_{i=0}^{n-1} E_i^c)\right)\\
&=\left(E_m \cap (\bigcap_{i=0, i \neq n}^{m-1} E_i^c)\right) \cap \emptyset \cap \left(\cap (\bigcap_{i=0}^{n-1} E_i^c)\right)\\
&= \emptyset \;. 
\end{align*}

If $x \in \bigcupdot F_n$, then $x \in F_n$ for some $n \geq 1$ and $x \in E_n \backslash (\bigcup_{i=0}^{n-1} E_i) \subset E_n \subset \bigcup E_n$. So $x \in \bigcup E_n$. If $x \in \bigcup E_n$, then $x \in E_n$ for some $n\geq 1$ ($x$ cannot be an element of $E_0$). If $x \in F_n = E_n \backslash (\bigcup_{i=0}^{n-1} E_i)$ then $x \in \bigcupdot F_n$. Otherwise this means $x \in E_k$ with $1\leq k \leq n-1$. Then if $x \in F_k = E_k \backslash (\bigcup_{i=0}^{k-1} E_i)$ then $x \in \bigcupdot F_n$. We can repeat this process until either $x \in F_k$ for some $k\geq 2$ or conclude that $x \in F_1= E_1$. In any case, we have $x \in \bigcupdot F_n$. These two arguments show that $\bigcup_{n=1}^\infty E_n = \bigcup_{n=0}^\infty E_n= \bigcupdot_{n=1}^\infty F_n$. \\

Note that $F_n \subset E_n$ for all $n\in \mathbb{N}$. By  Problem 1 $m(F_n) \leq m(E_n)$ for all $n \in \mathbb{N}$. Therefore,

$$m(\bigcup E_n) = m(\bigcupdot F_n) = \sum m(F_n) \leq \sum m(E_n) \;.$$

{\bf Problem 3} If there is a set $A \in \mathfrak{M}$ such that $m(A) < \infty$, then $m(\emptyset) = 0$. \\

Suppose $m(A) = \in [0,\infty)$. Define the sequence $(E_n)$ as $E_1 = A$ and $E_n = \emptyset$ for $n \geq 2$. Then $(E_n)$ is a sequence of disjoint sets and $A = \bigcupdot E_n$. Using $M_2$,

$$m(A) = m(\bigcupdot E_n) = \sum_{n=1}^\infty E_n = m(A) + \sum_{n=2}^\infty m(\emptyset)\;.$$

If $m(\emptyset) \neq 0$ then $m(\emptyset ) >0$ and $m(A) = m(A) + \sum_{n=2}^\infty m(\emptyset) = \infty$, which contradicts the assumption that $m(A)$ is finite. So it must be the case that $m(\emptyset) = 0$. \\

{\bf Problem 4} For $E \in \mathcal{P}(\mathbb{R})$, define $n(E) = \infty$ if $E$ is an infinite set and $n(E) = |E|$ if $E$ is a finite set (where $|E|$ is the number of elements in $E$). Show that $n$ is a countably additive set function that is translation invariant and defined for all sets of real numbers. This measure is called the {\bf counting measure}. \\

The set $\mathcal{P}(\mathbb{R})$ is a $\sigma$ algebra and $n: \mathcal{P}(\mathbb{R}) \rightarrow [0,\infty]$ since for any set $E$ of real numbers, the number of elements in $E$ is always nonnegative. Let $(E_n)$ be a sequence of disjoint sets of real numbers. Either at least one of the sets in the sequence $(E_n)$ is infinite or all sets in the sequence contain finitely many elements. If all sets in the sequence are finite, either all but a finite number of sets are empty or there is an infinite number of nonempty sets in the sequence.\\

If any set $E_k$ in the sequence is infinite, then $\bigcupdot E_n \supset E_k$ must also be infinite. Then $M_2$ is satisfied as $\infty = n(E_k) = n(\bigcupdot E_n) = \sum n(E_n) \geq n(E_k) = \infty \implies n(\bigcupdot E_n) = \sum n(E_n)$.\\

If all sets $E_k$ in the sequence $(E_n)$ are finite consider the two possible cases: (i) all but a finite number of sets are empty, (ii) there are infinitely many nonempty sets. If it is the case that all but a finite number of sets are empty, there is an $N \in \mathbb{N}$ such that $|E_n| \in [0,\infty) \forall n \leq N$ and $|E_n| = |\emptyset| = 0 \forall n>N$. Then $\bigcupdot_{n=1}^\infty E_n = \bigcupdot_{n=1}^N E_n$ is the union of a finite number of disjoint sets each containing a finite number of elements. Then the number of elements in $\bigcupdot_{n=1}^N E_n$ is the sum of the number of elements in each $E_n$; that is $n(\bigcupdot_{n=1}^\infty E_n) = |\bigcupdot_{n=1}^N E_n| = \sum_{n=1}^\infty |E_n| = \sum_{n=1}^N n(E_n) = \sum_{n=1}^\infty m(E_n)$. If it is the case that an infinite number of sets are nonempty, then the assumption that the sets in the sequence $(E_n)$ are disjoint means that $\bigcupdot E_n$ must have infinitely many elements. For any $B \in \mathbb{N}$, we have at least $B+1$ sets from the sequence each containing at least one real number and since no elements can be common to multiple sets, these $B+1$ sets contribute at least $B+1$ elements to the union $\bigcupdot E_n$. Since $n(E_n) > 0$ for infinitely many sets $E_n$ and $n(E_n) \geq 0$ for all $n$, $\sum_{n=1}^\infty m(E_n) = \infty$. Therefore, $n(\bigcupdot E_n) = \infty = \sum n(E_n)$. The assumption that the sets in $(E_n)$ are disjoint is necessary since otherwise there is no guarantee that the union of sets in the sequence will have infinitely many elements: consider $E_n = \{1\}$ for all $n$.\\

Conclude that if $(E_n)$ is a sequence of disjoint sets of real numbers, $n(\bigcupdot E_n) = \sum n(E_n)$.\\

Let $E$ be a set of real numbers. If $E$ contains infinitely many elements, adding $y \in \mathbb{R}$ to each element does not change the number of elements in $E$. So $n(E+y) = n(E)$. Similarly, if $E$ contains finitely many elements, adding $y$ to each element of $E$ may change the elements in the set but not the total number of elements in the set so again $n(E+y) = n(E)$. The function $n$ is translation invariant. \\


\end{document}