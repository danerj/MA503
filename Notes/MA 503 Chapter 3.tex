\documentclass[a4paper]{article}

%% Language and font encodings
\usepackage[english]{babel}
\usepackage[utf8x]{inputenc}
\usepackage[T1]{fontenc}

%% Sets page size and margins
\usepackage[a4paper,top=3cm,bottom=2cm,left=3cm,right=3cm,marginparwidth=1.75cm]{geometry}

%% Useful packages
\usepackage{amsmath}
\usepackage{graphicx}
\usepackage[colorinlistoftodos]{todonotes}
\usepackage[colorlinks=true, allcolors=blue]{hyperref}
\usepackage{float}
\usepackage{enumerate}
\usepackage{subfig}
\setlength\parindent{0pt}
\usepackage{amssymb}



\makeatletter
\def\moverlay{\mathpalette\mov@rlay}
\def\mov@rlay#1#2{\leavevmode\vtop{%
   \baselineskip\z@skip \lineskiplimit-\maxdimen
   \ialign{\hfil$\m@th#1##$\hfil\cr#2\crcr}}}
\newcommand{\charfusion}[3][\mathord]{
    #1{\ifx#1\mathop\vphantom{#2}\fi
        \mathpalette\mov@rlay{#2\cr#3}
      }
    \ifx#1\mathop\expandafter\displaylimits\fi}
\makeatother

\newcommand{\cupdot}{\charfusion[\mathbin]{\cup}{\cdot}}
\newcommand{\bigcupdot}{\charfusion[\mathop]{\bigcup}{\cdot}}

\title{MA 503 : Lebesgue Measure and Integration}
\author{Dane Johnson}

\begin{document}
\maketitle

\section*{Chapter 3 : Lebesgue Measure}

\subsection*{1 Introduction}

{\bf Definition} Let $X$ be a set and $\mathfrak{M}$ a collection of subsets of $X$, that is $\mathfrak{M} \subset \mathcal{P}(X)$. Then $\mathfrak{M}$ is a {\bf $\sigma$ - algebra} if

($\Sigma_1$) $X \in \mathfrak{M}$;\\
($\Sigma_2$) $A \in \mathfrak{M} \implies A^c \in \mathfrak{M}$;\\
($\Sigma_3$) For any sequence of sets $(A_n)\subset \mathfrak{M}$, $\cup_{n=1}^\infty A_n \in \mathfrak{M}$.\\

Considering ($\Sigma_2$), ($\Sigma_1$) could instead require $\emptyset \in \mathfrak{M}$ or both $\emptyset, X \in \mathfrak{M}$. \\

{\bf Definition / Proposition} Let $\mathcal{F} \subset \mathcal{P}(X)$. There is a smallest $\sigma$-algebra $\mathcal{C}$ such that $\mathcal{F} \subset \mathcal{C}$. We call $\mathcal{C}$ the {\bf $\sigma$-algebra generated by the collection of sets $\mathcal{F}$}. \\

Proof: Let $\mathcal{C} = \bigcap\{\mathcal{A} : \mathcal{F}\subset \mathcal{A}; \mathcal{A} \text{ is a } \sigma \text{ algebra }\}$. Note that the $\mathcal{A}$ are collections of sets so that an intersection would yield the collection of sets common to all such $\mathcal{A}$.  Claim that (i) $\mathcal{C}$ is a $\sigma$-algebra, (ii) $\mathcal{F}\subset \mathcal{C}$ and (iii) $\mathcal{C}$ is the smallest $\sigma$-algebra containing $\mathcal{F}$.\\

Since $X \in \mathcal{A}$ for all $\mathcal{A}$ used in the intersection, $X \in \mathcal{C}$. Let $S \in \mathcal{C}$ Then $S \in \mathcal{A}$ for all $\mathcal{A}$. Since all $\mathcal{A}$ are $\sigma$-algebras $S^c \in A$ for all $\mathcal{A}$. So $S^c \in \mathcal{C}$. If $(S_n)$ is a sequence of sets in $\mathcal{C}$, then $(S_n)\subset \mathcal{A}$ for all $\mathcal{A}$ and so $\bigcup S_n \in \mathcal{A}$ for all $\mathcal{A}$. Thus $\bigcup S_n \in \mathcal{C}$.\\

Let $S \in \mathcal{F}$ ($S$ a set in the set of sets $\mathcal{F}$). Then since $\mathcal{F}\subset \mathcal{A}$ for every $\mathcal{A}$ used in the intersection, $S \in \mathcal{A}$ for all $\mathcal{A}$ and so $S \in \mathcal{C}$. Thus $\mathcal{F}\subset \mathcal{C}$. \\

We have seen that $\mathcal{C}$ is a $\sigma$-algebra that contains $\mathcal{F}$. Suppose $\mathcal{D}$ is any $\sigma$-algebra that $\mathcal{F}\subset \mathcal{D}$. Then $\mathcal{D} \in \{\mathcal{A} : \mathcal{F}\subset \mathcal{A}; \mathcal{A} \text{ is a } \sigma \text{ algebra }\}$ by definition of this set and therefore $\mathcal{C} = \bigcap \{\mathcal{A} : \mathcal{F}\subset \mathcal{A}; \mathcal{A} \text{ is a } \sigma \text{ algebra }\} \subset \mathcal{D}$. That is $\mathcal{C}$ is an admissible $\sigma$-algebra and is a subset of any other admissible $\sigma$-algebra. In this sense $\mathcal{C}$ is the smallest $\sigma$-algebra containing $\mathcal{F}$.\\

{\bf Definition} The set $\mathcal{B}$ is the $\sigma$-algebra generated by the open sets in $\mathbb{R}$. The elements in $\mathcal{B}$ are called Borel sets. \\


{\bf Unspecified Homework 5 Problem} For $\mathcal{F} = \{S \subset \mathbb{R} : S \text{ is finite }\}$ find the smallest $\sigma$-algebra containing $\mathcal{F}$ as a subcollection. To say that a $\sigma$-algebra contains $\mathcal{F}$ as a subcollection means that the $\sigma$-algebra contains all of the sets in $\mathcal{F}$, which is itself a set of sets but not necessarily a $\sigma$-algebra. \\

Let $\sigma(\mathcal{F})$ denote the $\sigma$-algebra generated by $\mathcal{F}$. By the Definition / Proposition above $\sigma(\mathcal{F})$ is the smallest $\sigma$-algebra containing $\mathcal{F}$ as a subcollection. It will be shown that $\sigma(\mathcal{F}) = \{A \subset \mathbb{R} : A \text{ is countable or } A^c \text{ is countable }\}$. Here we take the definition of countable to mean finite or countably infinite. Sometimes the definition of countable is taken to mean specifically the case of countably infinite but that is not the definition to be considered here.\\

Proof: We will prove that $\sigma(\mathcal{F})$ as defined is indeed a $\sigma$-algebra with $\mathcal{F} \subset\sigma(F)$ and $\sigma(\mathcal{F})$ is the smallest $\sigma$-algebra containing $\mathcal{F}$ as a subcollection. Since these claims have not been established yet, let $\mathcal{C} = \{A \subset \mathbb{R} : A \text{ is countable or } A^c \text{ is countable }\}$.\\

First the claim that $\mathcal{C}$ is a $\sigma$-algebra will be verified. Since $\mathbb{R}^c = \emptyset$ and the empty set is countable as $|\emptyset| = 0$, $\mathbb{R} \in \mathcal{C}$. If $C \in \mathcal{C}$, then either $C$ is countable or $C^c$ is countable. If $C$ is countable, then since $(C^c)^c = C$, $C^c \in \mathcal{C}$ (whether or not $C^c$ is countable, the fact that $(C^c)^c$ is countable is enough to show that $C^c \in \mathcal{C}$). If $C$ is not countable, the assumption that $C\in \mathcal{C}$ means that $C^c$ must be countable. Then $C^c \in \mathcal{C}$. Suppose $(C_n)$ is a sequence of sets with $C_n \in \mathcal{C}$ for each $n \in \mathbb{N}$. Either all $C_n$ in the sequence $(C_n)$ are countable or there is at least one uncountable set $C_k$ in the sequence. If all $C_n$ are countable, then the countable union of countable sets $\bigcup_{n \in \mathbb{N}} C_n$ is also countable so $\bigcup_{n \in \mathbb{N}} C_n \in \mathcal{C}$. If it is not the case that $C_n$ is countable for all $n$, then there is at least one set uncountable set $C_{k}$ in the sequence. Since $C_k \in \mathcal{C}$ and $C_k$ is not countable, $C_k^c$ must be countable. Then $\bigcup_{n\in \mathbb{N}} C_n \supset C_k$ is not countable but since $\left(\bigcup_{n \in \mathbb{N}} C_n\right)^c = \bigcap_{n\in \mathbb{N}} C_n^c \subset C_k^c$ and $C_k^c$ is countable, $\left(\bigcup_{n \in \mathbb{N}} C_n\right)^c$ is countable. This means that $\bigcup_{n \in \mathbb{N}} C_n \in \mathcal{C}$. Conclude that $\mathcal{C}$ is a $\sigma$-algebra. \\

Next the claim that $\mathcal{F} \subset \mathcal{C}$ will be verified. Let $F \in \mathcal{F}$. Then $F$ is a finite set. A finite set is countable under our definition of countability. Therefore $F \in \mathcal{C}$. Since $F \in \mathcal{F}$ was arbitrary, $\mathcal{F}\subset \mathcal{C}$.  \\

Finally the claim that $\mathcal{C}$ is the smallest $\sigma$-algebra containing $\mathcal{F}$ as a subcollection will be verified. Suppose that $\mathcal{C}'$ is a $\sigma$-algebra containing $F$ as a subcollection. We will show that $\mathcal{C}\subset \mathcal{C}'$. Suppose for contradiction that $\mathcal{C} \not\subset \mathcal{C}'$ and that $C \in \mathcal{C}$ but $C \not\in \mathcal{C}'$. Either $C$ is countable or $C^c$ is countable. If $C$ is countable, we may order the elements of $C$ as $C = \{c_1,...,c_n\}$ if $C$ is finite or $C = \{c_1,c_2,c_3,...\}$ if $C$ is countably infinite. In the case that $C$ is finite then $C \in \mathcal{F}$ but $C \not\in \mathcal{C}'$, which contradicts the assumption that $\mathcal{C}'$ contains $\mathcal{F}$ as a subcollection. If $C$ is countably infinite, note that the singletons $\{c_i\}, i \in \mathbb{N}$ are all finite (and so elements of $\mathcal{F}$) and since $\mathcal{C}'$ is a $\sigma$-algebra containing every set in $\mathcal{F}$, $C = \bigcup_{i\in \mathbb{N}} c_i \in \mathcal{C}'$, which contradicts the assumption that $C \not\in \mathcal{C}'$. If $C$ is not countable then $C^c$ is countable. By similar reasoning, it follows that $C^c \in \mathcal{C}'$. But since $\mathcal{C}'$ is a $\sigma$-algebra, this again gives the contradiction that $C \in \mathcal{C}'$. The assumption that $\mathcal{C} \not\subset \mathcal{C}'$ leads in any case to a contradiction, so $\mathcal{C} \subset \mathcal{C}'$ for any $\sigma$-algebra $\mathcal{C}'$ that contains $\mathcal{F}$ as a subcollection. In this sense $\mathcal{C}$ is the minimal (or smallest) $\sigma$-algebra that contains $\mathcal{F}$ as a subcollection.\\

Since $\mathcal{C} =\{A \subset \mathbb{R} : A \text{ is countable or } A^c \text{ is countable }\}$ is a $\sigma$-algebra containing $\mathcal{F}$ as a subcollection and $\mathcal{C} \subset \mathcal{C}'$ for any $\sigma$-algebra $\mathcal{C}'$ that contains $\mathcal{F}$ as a subcollection, $\mathcal{C}$ is the smallest $\sigma$-algebra containing $\mathcal{F}$ as a subcollection and we may write $\mathcal{C} = \sigma(\mathcal{F})$ to specify that this is the $\sigma$-algebra generated by $\mathcal{F}$. \\



{\bf Definition} Let $\mathfrak{M}$ be a $\sigma$ algebra  of sets (or real numbers). A function $m$ is called a {\bf countably additive measure} if\\

($M_1$) $m : \mathfrak{M}\rightarrow [0,+\infty]$\\
($M_2$) For any  sequence of disjoint sets $(E_n) \subset \mathfrak{M}$, $m(\bigcupdot E_n) = \sum m(E_n)$.\\

In problems 1 - 3 let $m$ be a countably additive measure  defined for all sets in a $\sigma$ algebra $\mathfrak{M}$. Whether this omission is an error or intentional, there is no evidence that $m(\emptyset) := 0$, so it is necessary to prove some lemmas to handle this inconvenience. Problem 3 essentially handles this but I was unsure how else to complete Problem 1 without using the measure of the empty set in some way.  \\

{\bf Lemma 1} $m(\emptyset) = 0$ or $m(\emptyset) = \infty$.\\

Proof: Suppose $ m(\emptyset) = \alpha \in (0,\infty)$. Then since $\emptyset = \bigcupdot_{i = 1}^\infty \emptyset$, we obtain $\alpha = m(\emptyset) = m(\bigcupdot_{i = 1}^\infty \emptyset) = \sum_{i=1}^\infty m(\emptyset) = \sum_{i=1}^\infty \alpha = \infty$. This contradicts the assumption that $\alpha < \infty$. Therefore $m(\emptyset) = 0$ or $m(\emptyset) = \infty$ \\

{\bf Lemma 2} If $m(\emptyset) = \infty$, $m(A) = \infty$ for each $A$ in $\mathfrak{M}$. \\

The sequence $(E_n)$ with $E_1 = A$, $E_n = \emptyset$ for $n\geq 2$ is a disjoint sequence. By $M_2$,
$$m(A) = m\left(\bigcupdot E_n\right) = \sum E_n = m(A) + \sum_{n\geq 2} E_n  \geq 0 + \sum_{n\geq 2} E_n = \infty \implies m(A) = \infty \;.$$


{\bf Problem 1} If $A$ and $B$ are two sets in $\mathfrak{M}$ with $A \subset B$, then $m(A) \leq m(B)$. This property is called {\bf monotonicity}.\\


Proof: If $m(\emptyset) = \infty$ the inequality holds by Lemma 2. Otherwise assume this is not the case so that $m(\emptyset) = 0$ by Lemma 2. Since $A\subset B$, $A\cup B = B$. The sets $A$ and $B\backslash A$ are disjoint since $A \cap (B \backslash A) = A \cap (B \cap A^c) = (A\cap A^c) \cap B = \emptyset \cap B = \emptyset$ while $A \cupdot (B\backslash A) = A\cupdot(B \cap A^c) = (A\cup B) \cap (A \cupdot A) = (A\cup B) \cap \mathbb{R} = A\cup B = B$. Also, $\emptyset \cap A = \emptyset$, $\emptyset \cap  (B\backslash A) = \emptyset$, and $\emptyset\cap \emptyset = \emptyset$. Define the sequence $(E_n)$ by $E_1 = A$, $E_2 = B\backslash A$ and $E_n = \emptyset$ for $n \geq 3$. Then $(E_n)$ is a sequence of disjoint sets and by $M_2$,
$$0 \leq m(B) = m(A\cupdot (B \backslash A) = m(\bigcupdot E_n) = \sum m(E_n) = m(A) + m(B\backslash A) + 0 + 0 + ... = m(A) + m(B\backslash A) \;.$$

This shows that $m(A) + m(B\backslash A) = m(B)$. Since $m(B\backslash A) \geq 0$, $m(B) \geq m(B) - m(B\backslash A) = m(A)$. Therefore $m(A) \leq m(B)$. \\


{\bf Problem 2} Let $(E_n)$ be a sequence of sets in $\mathfrak{M}$. Then $m(\bigcup E_n) \leq \sum m(E_n)$. This property is called {\bf countable subadditivity}. \\

Proof: If $m(\emptyset) = \infty$ the inequality holds immediately. Assume this is not the case so that $m(\emptyset) = 0$. Define $E_0 : = \emptyset$ for notational consistency and the sequence $(F_n)$ by
\begin{align*}
F_1 &= E_1 \backslash E_0 \\
F_2 &= E_2 \backslash F_1 = E_2 \backslash (E_0 \cup E_1)\\
F_3 &= E_3 \backslash (E_0 \cup E_1 \cup E_2) \\
F_4 &= E_4 \backslash (E_0 \cup E_1 \cup E_2 \cup E_3) \\
&... \\
F_n &= E_n \backslash (\bigcup_{i=0}^{n-1} E_i)
\end{align*}

Suppose $m\neq n$ and without loss of generality that $n < m$.

\begin{align*}
F_m \cap F_n &= \left(E_m \cap (\bigcup_{i=0}^{m-1} E_i)^c\right) \cap \left(E_n \cap (\bigcup_{i=0}^{n-1} E_i)^c\right)\\
&=\left(E_m \cap (\bigcap_{i=0}^{m-1} E_i^c)\right) \cap \left(E_n \cap (\bigcap_{i=0}^{n-1} E_i^c)\right)\\
&=\left(E_m \cap (E_n^c \cap \bigcap_{i=0, i \neq n}^{m-1} E_i^c)\right) \cap \left(E_n \cap (\bigcap_{i=0}^{n-1} E_i^c)\right)\\
&=\left(E_m \cap (\bigcap_{i=0, i \neq n}^{m-1} E_i^c)\right) \cap (E_n^c \cap E_n) \cap \left(\cap (\bigcap_{i=0}^{n-1} E_i^c)\right)\\
&=\left(E_m \cap (\bigcap_{i=0, i \neq n}^{m-1} E_i^c)\right) \cap \emptyset \cap \left(\cap (\bigcap_{i=0}^{n-1} E_i^c)\right)\\
&= \emptyset \;. 
\end{align*}

If $x \in \bigcupdot F_n$, then $x \in F_n$ for some $n \geq 1$ and $x \in E_n \backslash (\bigcup_{i=0}^{n-1} E_i) \subset E_n \subset \bigcup E_n$. So $x \in \bigcup E_n$. If $x \in \bigcup E_n$, then $x \in E_n$ for some $n\geq 1$ ($x$ cannot be an element of $E_0$). If $x \in F_n = E_n \backslash (\bigcup_{i=0}^{n-1} E_i)$ then $x \in \bigcupdot F_n$. Otherwise this means $x \in E_k$ with $1\leq k \leq n-1$. Then if $x \in F_k = E_k \backslash (\bigcup_{i=0}^{k-1} E_i)$ then $x \in \bigcupdot F_n$. We can repeat this process until either $x \in F_k$ for some $k\geq 2$ or conclude that $x \in F_1= E_1$. In any case, we have $x \in \bigcupdot F_n$. These two arguments show that $\bigcup_{n=1}^\infty E_n = \bigcup_{n=0}^\infty E_n= \bigcupdot_{n=1}^\infty F_n$. Alternatively,

\begin{align*}
\bigcupdot F_n &= \bigcup_{n = 1}^\infty \left(E_n \cap (\bigcup_{i = 1}^{n-1} E_i)^c \right)\\
&=\bigcup_{n = 1}^\infty \left(E_n \cap (\bigcap_{i = 1}^{n-1} E_i^c) \right)\\
\end{align*} 

Note that $F_n \subset E_n$ for all $n\in \mathbb{N}$. By  Problem 1 $m(F_n) \leq m(E_n)$ for all $n \in \mathbb{N}$. Therefore,

$$m(\bigcup E_n) = m(\bigcupdot F_n) = \sum m(F_n) \leq \sum m(E_n) \;.$$

{\bf Problem 3} If there is a set $A \in \mathfrak{M}$ such that $m(A) < \infty$, then $m(\emptyset) = 0$. \\

Suppose $m(A) = \in [0,\infty)$. Define the sequence $(E_n)$ as $E_1 = A$ and $E_n = \emptyset$ for $n \geq 2$. Then $(E_n)$ is a sequence of disjoint sets and $A = \bigcupdot E_n$. Using $M_2$,

$$m(A) = m(\bigcupdot E_n) = \sum_{n=1}^\infty E_n = m(A) + \sum_{n=2}^\infty m(\emptyset)\;.$$

If $m(\emptyset) \neq 0$ then $m(\emptyset ) >0$ and $m(A) = m(A) + \sum_{n=2}^\infty m(\emptyset) = \infty$, which contradicts the assumption that $m(A)$ is finite. So it must be the case that $m(\emptyset) = 0$. \\

{\bf Problem 4} For $E \in \mathcal{P}(\mathbb{R})$, define $n(E) = \infty$ if $E$ is an infinite set and $n(E) = |E|$ if $E$ is a finite set (where $|E|$ is the number of elements in $E$). Show that $n$ is a countably additive set function that is translation invariant and defined for all sets of real numbers. This measure is called the {\bf counting measure}. \\

The set $\mathcal{P}(\mathbb{R})$ is a $\sigma$ algebra and $n: \mathcal{P}(\mathbb{R}) \rightarrow [0,\infty]$ since for any set $E$ of real numbers, the number of elements in $E$ is always nonnegative. Let $(E_n)$ be a sequence of disjoint sets of real numbers. Either at least one of the sets in the sequence $(E_n)$ is infinite or all sets in the sequence contain finitely many elements. If all sets in the sequence are finite, either all but a finite number of sets are empty or there is an infinite number of nonempty sets in the sequence.\\

If any set $E_k$ is the sequence is infinite, then $\bigcupdot E_n \supset E_k$ must also be infinite. Then $M_2$ is satisfied as $\infty = n(E_k) = n(\bigcupdot E_n) = \sum n(E_n) \geq n(E_k) = \infty \implies n(\bigcupdot E_n) = \sum n(E_n)$.\\

If all sets $E_k$ in the sequence $(E_n)$ are finite consider the two possible cases: (i) All but a finite number of sets are empty, (ii) There are infinitely many nonempty sets. If it is the case that all but a finite number of sets are empty, there is an $N \in \mathbb{N}$ such that $|E_n| \in [0,\infty) \forall n \leq N$ and $|E_n| = |\emptyset| = 0 \forall n>N$. Then $\bigcupdot_{n=1}^\infty E_n = \bigcupdot_{n=1}^N E_n$ is the union of a finite number of disjoint sets each containing a finite number of elements. Then the number of elements in $\bigcupdot_{n=1}^N E_n$ is the sum of the number of elements in each $E_n$; that is $n(\bigcupdot_{n=1}^\infty E_n) = |\bigcupdot_{n=1}^N E_n| = \sum_{n=1}^\infty |E_n| = \sum_{n=1}^N n(E_n) = \sum_{n=1}^\infty m(E_n)$. If it is the case that an infinite number of sets are nonempty, then the assumption that the sets in the sequence $(E_n)$ are disjoint means that $\bigcupdot E_n$ must have infinitely many elements. For any $B \in \mathbb{N}$, we have at least $B+1$ sets from the sequence each containing at least one real number and since no elements can be common to multiple sets, these $B+1$ sets contribute at least $B+1$ elements to the union $\bigcupdot E_n$. Since $n(E_n) > 0$ for infinitely many sets $E_n$ and $n(E_n) \geq 0$ for all $n$, $\sum_{n=1}^\infty m(E_n) = \infty$. Therefore, $n(\bigcupdot E_n) = \infty = \sum n(E_n)$. The assumption that the sets in $(E_n)$ are disjoint is necessary since otherwise there is no guarantee that the union of sets in the sequence will have infinitely many elements: consider $E_n = \{1\}$ for all $n$.\\

Conclude that if $(E_n)$ is a sequence of disjoint sets of real numbers, $n(\bigcupdot E_n) = \sum n(E_n)$.\\

Let $E$ be a set of real numbers. If $E$ contains infinitely many elements, adding $y \in \mathbb{R}$ to each element does not change the number of elements in $E$. So $n(E+y) = n(E)$. Similarly, if $E$ contains finitely many elements, adding $y$ to each element of $E$ may change the elements in the set but not the total number of elements in the set so again $n(E+y) = n(E)$. The function $n$ is translation invariant. \\

\subsection*{2 Outer Measure}

{\bf Definition} For each set $A$ of real numbers consider the set of countable sets $\{I_n\}$ of open intervals such that $A \subset \bigcup I_n$. For each set of intervals $\{I_n\}$, consider the sum of the lengths of the intervals in the set. Since these lengths are nonnegative this sum is uniquely defined independent of the ordering of the intervals. We define the {\bf outer measure} of $A$, $m^*(A)$, as the infimum of all such sums.

$$m^*(A) = \text{inf}\{\sum_n l(I_n) : A \subset \bigcup I_n\} \;.$$

{\bf Note} Since any collection of intervals covers the empty set, taking $\{(0,1/n)\}$ for each natural number $n$ shows $m^*(\emptyset) \leq 0$ and since the sum of lengths of any collection of intervals is nonnegative that $m^*(\emptyset) = 0$. If $A\subset B$, then $\{\sum_n l(I_n) : B \subset \bigcup I_n\} \subset \{\sum_n l(I_n) : A \subset \bigcup I_n\}$ since any collection of intervals that covers $B$ must also cover $A$. For sets $C \subset D$ of real numbers bounded below, $\text{inf } D \leq \text{inf } C$. Since $\{\sum_n l(I_n) : B \subset \bigcup I_n\}$ and $\{\sum_n l(I_n) : A \subset \bigcup I_n\}$ are nonempty and bounded below by 0, $m^*(A) \leq m^*(B)$.\\

{\bf Proposition 1} The outer measure of an interval is its length. \\

Proof: Begin with the case of a closed finite interval $[a,b]$. Since $I_\epsilon = (a-\epsilon, b + \epsilon)$ contains $[a,b]$ for each $\epsilon > 0$ and $m^*([a,b]) \leq \sum l(I_\epsilon) = l(I_\epsilon) = b-a + 2\epsilon$ (there is only one interval in the cover). This holds for every $\epsilon$ so $m^*([a,b]) \leq b-a$. If we can show that $\sum l(I_n) \geq b-a$ for every $\{I_n\}$ covering $[a,b]$, then by definition of greatest lower bound, $m^*([a,b]) \geq b-a$ and we can conclude that $m^*([a,b]) = b-a$. For any collection $\{I_n\}$, there is a finite subcollection that covers $[a,b]$ by Heine - Borel Theorem. The sum of the lengths of the intervals in the finite collection is less than or equal to the sum of the lengths in the possibly infinite collection, so it suffices to consider only finite collections of open intervals that cover $[a,b]$. Let $\{I_n\}$ be such a collection of open intervals. Since $a \in \bigcup I_n$, $a \in I_n$ for some $n$. Write $a \in (a_1,b_2)$ with $a_1<a<b_1$. If $b_1 > b$, then $\sum l(I_n) \geq l((a_1,b_1)) > b-a$ and we are done. If $b_1 \leq b$ then $b_1 \in [a,b]$ and since $b_1 \not\in (a_1,b_2)$, $b_1 \in (a_2,b_2)$ for some other interval in the collection. Then $a_2<b_1<b_2$. If $b_2 > b$ then we are done since

\begin{align*}
\sum l(I_n) &\geq (b_2 - a_2) + (b_1 - a_1)\\
&= b_2 -(a_2 - b_1) - a_1 \\
&> b_2 - a_1 \quad \text{because } -(a_2 - b_1) > 0 \\
&> b-a \quad \text{because } b_2>b, \; a_1<a \;.
\end{align*}

Otherwise if $b_2\leq b$, $b_2 \in (a_3,b_3)$ for one of the intervals in the collection. Suppose there are $N$ intervals in the collection. If we find for some $k<N$ that $b_k > b$ then we are done since

\begin{align*}
\sum l(I_n) &\geq (b_k - a_k) + (b_{k-1} - a_{k-1}) + ...+(b_2 - a_2) + (b_1 - a_1)\\
&= b_k -(a_{k} - b_{k-1}) - ...(a_2 - b_1) - a_1 \\
&> b_k - a_1 \quad \text{because for } j=2,...,k, \quad  -(a_{j} - b_{j-1}) > 0  \\
&> b-a \quad \text{because } b_k>b, \; a_1<a \;.
\end{align*}

Otherwise $b_{N-1} \in (a_n,b_N)$. If we have reached this point, then $b_k \leq b$ for each $k$. Since the collection of intervals must cover $[a,b]$, we must have $b_n > b$ and by similar computation to before,
$$\sum l(I_n) = (b_N - a_N) + ... + (b_1 - a_1) = b_N - (a_N - b_{N-1}) - ... - (a_2 -b_1) - a_1 > b_N - a_1 > b-a \;.$$

The collection $\{I_n\}$ was arbitrary and we showed that $\sum l(I_n) > b-a \implies \sum l(I_n) \geq b-a$. Therefore, $\sum l(I_n) \geq b-a$ for every finite collection of open intervals that covers $[a,b]$. Any infinite collection of open intervals that covers $[a,b]$ has a finite subcollection that covers $[a,b]$ with the sum of lengths less than or equal to the infinite collection. Therefore, for any countable collection of open intervals that covers $[a,b]$, $\sum l(I_n) \geq b-a$. Conclude that $m^*([a,b]) = b-a$.\\

If $I$ is any finite interval (open, closed, half open/ half closed), then given $\epsilon > 0$ there is a closed interval $J\subset I$ such that $l(J) + \epsilon > l(I)$. This can be shown by cases, but consider an example to see how to different cases could be proven with small adjustments. Let $I = [a,b)$ and $\epsilon > 0$. Take $J = [a,b-\epsilon/2]$. Then $J \subset I$ and $l(J) + \epsilon = b-\epsilon/2 - a + \epsilon = b-a + \epsilon/2 > b-a = l(I)$. The point is to find for an arbitrary finite interval $I$ a closed and bounded interval $J$ such that $J$ has the same length as $I$ ($I = J$) or is contained within $I$ and only slightly smaller length so that the intervals are approximately the same length. Since $J$ is closed and bounded, we can use the previous result. The closure of $I$ is also closed and bounded and we have seen that since $J\subset I$, $m^*(J) \leq m^*(I)$. The length of any finite interval of any sort is the difference of the greater endpoint and the lower endpoint. Thus for every $\epsilon>0$ we can find $J$ such that 
$$l(I) - \epsilon < l(J) = m^*(J) \leq m^*(I) \leq m^*(\overline{I}) = l(\overline{I}) = l(I) \;.$$

This implies $l(I) \leq m^*(I) \leq l(I)$ and so $m^*(I) = l(I)$. \\

If $I$ is an infinite interval, then $l(I) = \infty$. Also, for each $\Delta \in \mathbb{R}$ there is a closed and bounded interval $J$ with $l(J) = \Delta$ and $J\subset I$. Then $\Delta = l(J) = m^*(J) \leq m^*(I)$. Since $m^*(I) \geq \Delta$ for each $\Delta$, $m^*(I) = \infty = l(I)$. \\

{\bf Proposition 2} Let $A_n$ be a countable collection of sets of real numbers.
$$m^*\left(\bigcup A_n\right) \leq \sum m^*(A_n) \;.$$

Proof: If $m^*(A_n) = \infty$ for any $A_n$, then $\infty = m^*(A_n) \leq m^*\left(\bigcup A_n \right) \leq \infty$ and $\infty  = m^*(A_n) \leq \sum m^*(A_n) \leq \infty$ so the inequality holds. Otherwise assume $m^*(A_n) \in [0,\infty)$ for all $A_n$. For each $A_n$, given $\epsilon > 0$, $m^*(A_n) + 2^{-n}\epsilon$ is not a lower bound of the set $\{\sum_n l(I_n) : A \subset \bigcup I_n\}$. So there is a countable collection of open intervals $\{I_{n,i}\}$ such that $A_n \subset \bigcup_{i} I_{n,i}$ and $m^*(A_n) \leq \sum_{i} l(I_{n,i}) < m^*(A_n) + 2^{-n}\epsilon$. Each $A_n \subset \bigcup A_n$ and the countable collection of countable collections $\{I_{n,i} : n \in \mathbb{N}, i \in \mathbb{N}\}$ of collections corresponding to each $A_n$ is countable and $\bigcup A_n \subset \bigcup_{n} \bigcup_{i} I_{n,i} = \bigcup_{n,i} I_{n,i}$. To see this note that if $x \in \bigcup A_n$, $x \in A_n$ for some $n$ and so $x \in I_{n,i}$ for some interval and so $x \in \bigcup_{n,i} I_{n,i}$. Therefore, 
$$m^*\left(\bigcup A_n \right) \leq \sum_{n} \sum_{i} I_{n,i} < \sum_{n} \left(m^*(A_n) + 2^{-n}\epsilon\right) = \left(\sum_{n} m^*(A_n)\right) + \epsilon\;.$$

This holds for each $\epsilon > 0$ so $m^*\left(\bigcup A_n\right) \leq  \sum m^*(A_n)$. \\

{\bf Corollary Stringbean} For sets $A$ and $B$, $m^*(A\cup B) \leq m^*(A) + m^*(B)$.

Proof: Set $C_1 = A$, $C_2 = B$, and $C_n = \emptyset$ for $n \geq 3$. Then by Proposition 2, $m^*(A\cup B) = m^*\left(\bigcup C_n\right) \leq \sum m^*(D_n) = m^*(A) + m^*(B)$. \\

{\bf Corollary 3} If $A$ is countable, $m^*(A) = 0$. \\

Proof: Let $\epsilon >0$ and for each $a_n \in A$, define the open interval $I_n = (a-\epsilon/2^{n+1}, a+ \epsilon/2^{n+1})$ so that $l(I_n) = \epsilon/2^n$. Then $\bigcup_{n} \{a_n\} = A$ and since $\{a_n\} \subset I_n$, $m^*(\{a_n\}) \leq m^*(I_n)$. By Proposition 2,
$$m^*(A) = m^*\left(\bigcup \{a_n\}\right) \leq \sum m^*(\{a_n\}) \leq \sum m^*(I_n) = \sum l(I_n) = \epsilon/2^n = \epsilon \;.$$

Since $m^*(A) \leq \epsilon$ for each $\epsilon$, $m^*(A) < \epsilon $ for each $\epsilon$, which implies $m^*(A) = 0$. \\

{\bf Corollary 4} The set $[0,1]$ is not countable. \\

Proof: By contrapositive restatement of Corollary 3, if $A$ is a set with $m^*(A) \neq 0$, $A$ is uncountable. Since $[0,1]$ is an interval, $m^*([0,1]) = l([0,1]) = 1\neq 0$ by Proposition 1. Therefore $[0,1]$ is not countable. \\


{\bf Problem 5} Let $A$ be a set of rational numbers between 0 and 1, and let $\{I_n\}$ be a finite collection of open intervals covering $A$. Prove that $\sum l(I_n) \geq 1$. \\

Let $A = \mathbb{Q}\cap [0,1]$ and $\{I_n\}$ a finite collection of $N$ open intervals such that $A\subset \bigcup_{n=1}^N I_n$. For each $n$, denote $I_n = (a_n,b_n)$ for $a_n,b_n \in \mathbb{R}$. Assume, relabeling the finite number indices if necessary, that $a_1 \leq ... \leq a_N$. For $i\neq j$ and $a_i \leq a_j$, if $b_i \geq b_j$, then $(a_j,b_j) \subset (a_i,b_i)$ and we can remove the interval $(a_j,b_j)$ from the collection without losing coverage of $A$. Also, if we have any intervals $(a_j,b_j)$ with either $a_j \geq 1$ or $b_j \geq 0$ then remove this interval as well as $(a_j,b_j) \cap A = \emptyset$. After removing $k$ such superfluous intervals, assume we have $N := N-k$ (possibly just reassigning $N$ to avoid introducing an extra letter to the proof) intervals $(a_1,b_1),...,(a_N,b_N)$ remaining with $A \subset \bigcup_{n = 1}^N I_n = \bigcup_{n =1}^N (a_n,b_n)$. If $a_1> 0 $, then since $a_1\leq a_j$ for each $j$, $[0,a_1)$ is not contained in any interval in the collection and so $A \not\subset \bigcup (a_n,b_n)$. So we must have $a_1 < 0$. Similarly, if $b_N < 1$, then since $b_N \geq b_j$ for each $j$ (this is a result of the process of removing superfluous intervals - if $b_j > b_N$ for $j < N$ then $a_j \leq a_N$ and $b_j > b_N$ $\implies (a_N,b_N) \subset (a_j, b_j)$, which means $(a_N,b_N)$ would have been removed), $A \not\subset \bigcup (a_n,b_n)$. So we must have $b_N > 1$.  If $N=1$, then

$$\sum l(I_n) \geq \sum_{n=1}^1 (a_n,b_n) = b_N - a_N > 1-0 = 1 \;.$$
Otherwise for $N < 1$, for each $n \in \{1,...,N-1\}$, $b_n \geq a_{n+1}$. Suppose instead that $b_n < a_{n+1}$. There is a rational number $r$ such that $r \in (b_n,a_{n+1}$. We know that $b_n > 0$ and $a_{n+1} < 1$ by the process used to remove unnecessary sets. Then $r \in A$ but $r \not in \bigcup (a_n,b_n)$. So $b_n \geq a_{n+1}$ for $n \in \{1,...,N-1\}$. Note that we can only have $b_n = a_{n+1}$ if $b_n = a_{n+1}$ is irrational, but that this consideration will not affect the inequalities given next. Therefore $0\leq  -(a_{n+1} - b_n$) and

\begin{align*}
\sum l(I_n) & \geq \sum_{n=1}^N l(I_n) \quad \text{ removing the superfluous intervals of nonnegative length}\\
&= \sum_{n=1}^N l((a_n,b_n)) \quad \text{we chose to denote } I_n = (a_n,b_n)\\
&= (b_N - a_N) + (b_{N-1} -a_{N-1}) + ... + (b_1 - a_1)\\
&= b_N - (a_{N}-b_{N-1}) - ... - a_1 \\
&\geq b_N - a_1 \quad \text{because } -(a_{n+1} - b_n) \geq 0,\quad  n \in \{1,...,N-1\}\\
&> 1 - 0 \quad \text{because } b_N > 1, \quad a_1 < 0\\
&= 1
\end{align*}

{\bf Definition} We say that $F \in F_\sigma$ if $F$ is a countable union of closed sets. We say that $G \in G_\delta$ if $G$ is a countable intersection of open sets ($\delta$ is for durchschnitt - durchschnitten means to "cut through" or "intersect" in German).\\ 

{\bf Proposition 5} Given any set $A$ and $\epsilon > 0$, there is an open set $\mathcal{O}$ such that $A\subset \mathcal{O}$ and $m^*(O) \leq m^*(A) + \epsilon$. There is a $G \in G_\delta$ such that $A \subset G$ and $m^*(A) = m^*(G)$. \\

{\bf Problem 6} Prove Proposition 5.\\

Proof: Let $A\subset \mathbb{R}$ and $\epsilon > 0$. We have seen that $m^*(A)$ is defined for any set of real numbers. First if $A = \emptyset$, then since $\emptyset$ is open and $\emptyset \subset \emptyset$ we have $m^*(\emptyset) = 0 < 0 + \epsilon = m^*(A) + \epsilon$. Otherwise if $A$ is such that $m^*(A) = \infty$, then since $\mathbb{R}$ is open and $A \subset \mathbb{R}$ by hypothesis, $m^*(A) = \infty =: \infty + \epsilon$ and $m^*(\mathbb{R})  = \infty$ so $m^*(\mathbb{R}) \leq m^*(A) + \epsilon$.\\

Assume that $A \subset \mathbb{R}$ such that $m^*(A) \in [0,\infty)$ but $A \neq \emptyset$. By definition of infimum, $m^*(A) + \epsilon$ is not a lower bound of the set $\{\sum l(I_n) : A \subset \bigcup I_n\}$ (where all collections $\{I_n\}$ are countable collections of open intervals). So there is a countable collection of open intervals $\{I_n\}$ such that $A \subset \bigcup I_n$ and $m^*(A) \leq \sum l(I_n) < m^*(A) + \epsilon$. Then $\mathcal{O} := \bigcup I_n$ is open as a union of open sets and 
\begin{align*}
m^*(\mathcal{O}) &= m^*\left(\bigcup I_n\right) \\
&\leq \sum m^*(I_n) \quad \text{ (Proposition 2)}\\
&= \sum l(I_n) \quad \text{ (Proposition 1)}\\
&< m^*(A) +\epsilon \;.
\end{align*}

Therefore $m^*(\mathcal{O}) < m^*(A) + \epsilon \implies
m^*(\mathcal{O}) \leq m^*(A) + \epsilon$.\\

Next we consider the claim that there is a countable intersection of open sets, $G$, such that $A \subset G$ and $m^*(A) = m^*(G)$. If $A = \emptyset$, take $G = \cap_{n \in \mathbb{N}} G_n$ with $G_n = \emptyset$ for all $n$. Then $A = \emptyset = G$ and $m^*(A) = 0 = m^*(G)$. If $A$ is such that $m^*(A) = \infty$, take $G = \cap_{n \in \mathbb{N}}$ with $G_n = \mathbb{R}$ for all $n$ so that $A \subset G = \mathbb{R}$ and $m^*(A) = \infty = m^*(G)$. \\

Assume that $A \subset \mathbb{R}$ such that $m^*(A) \in [0,\infty)$ and $A \neq \emptyset$. For each $n \in \mathbb{N}$, by the reasoning explained in the first part of the proof, there is a countable collection of open intervals $\{I_m^n\}_{m \in \mathbb{N}}$ such that $A \subset \bigcup_{m} I_m^n$ and $m^*(A) \leq m^*\left(\bigcup_{m} I_m^n \right) < m^*(A) + 1/n$. For notational convenience, denote $G_n = \bigcup_{m} I_m^n$. Then for each $n$, $G_n$ is open as a union of open sets. Since $A \subset G_n$ for each $n$, $A \subset G:= \bigcap_{n \in \mathbb{N}} G_n$ and $G$ is a countable intersection of open sets. For each $n \in \mathbb{N}$

\begin{align*}
m^*(A) &\leq m^*(G) \quad (\text{as } A \subset G)\\
&\leq m^*(G_n) \quad (\text{as } G \subset G_n)\\
&< m^*(A) + 1/n \;.
\end{align*}

Since this holds for each $n$, $m^*(A) \leq m^*(G) \leq m^*(A)$. Therefore, $G$ is a $G_\delta$ set, $A\subset G$, and $m^*(A) = m^*(G)$. \\

{\bf Problem 7} Prove that $m^*$ is translation invariant.\\

Let $A \subset \mathbb{R}$ and $y \in \mathbb{R}$.
If $A = \emptyset$, then $A+y = \emptyset$ and $m^*(A) = m^*(\emptyset) = m^*(A+y)$. \\

For $A,A+y \neq \emptyset$, let $s \in \{\sum l(I_n) : A \subset \bigcup I_n \}$, where each collection $\{I_n\}$ is a countable collection of open intervals ($s > 0$ since at least one of the intervals is a true open interval, i.e. not of the form $(x,x) = \emptyset)$). Then $s \in (0,\infty]$ and $s = \sum l(I_n)$ for some countable collection of open intervals $\{I_n := (a_n,b_n)\}$ such that $A \subset \bigcup I_n$. The countable collection of open intervals $\{I_n + y = (a_n + y, b_n + y)\}$ covers $A + y$. That is, $A + y \subset \{I_n + y = (a_n + y, b_n + y)\}$. To prove this let $x \in A+y$. Then $x = a+y$ for some $a \in A$. Since $A\subset \bigcup I_n$, there is an open interval in the collection $\{I_n\}$ such that $a \in I_j = (a_j,b_j)$. Since $a_j < a<b_j$, $a_j + y < a+ y < b_j + y$ so that $a+y \in (a_j + y, b_j + y) = I_j + y \subset \bigcup (I_n + y)$. Next note that $\sum l((I_n + y)) = \sum l((a_n + y, b_n+y)) = \sum b_n - a_n = \sum l((a_n,b_n)) = \sum l(I_n) = s$. Therefore $s = \sum l(I_n +y)$ for some countable collection of open intervals $\{I_n + y\}$ such that $A+y \subset \bigcup (I_n + y)$ and so $s \in \{\sum l(I_n + y) : A +y\subset \bigcup (I_n+y) \}$. Therefore $\{\sum l(I_n) : A \subset \bigcup I_n \} \subset \{\sum l(I_n + y) : A +y\subset \bigcup (I_n+y) \}$. Since these steps are reversible, we have similarly that $\{\sum l(I_n+y) : A +y\subset \bigcup (I_n+y) \} \subset \{\sum l(I_n) : A \subset \bigcup I_n \}$.\\

Therefore, $\{\sum l(I_n) : A \subset \bigcup I_n \} = \{\sum l(I_n+y) : A +y\subset \bigcup (I_n+y) \}$. This set is a set of real numbers bounded below by 0 and $m^*(A) = \text{inf}\{\sum l(I_n) : A \subset \bigcup I_n \} = \text{inf}\{\sum l(I_n + y) : A \subset \bigcup (I_n+y) \} = m^*(A+y)$. \\

{\bf Problem 8} Prove that if $m^*(A) = 0$, then $m^*(A\cup B) = m^*(B)$.\\

Suppose $m^*(A) = 0$. Since $B \subset A\cup B$, by the note made following the definition of outer measure $m^*(B) \leq m^*(A\cup B)$ and $m^*(\emptyset) = 0$. Define the sequence $(D_n)$ by $D_1 = A, D_2 = B$, and $D_n = \emptyset$ for $n \geq 3$. By Proposition 2,  $$m^*(A\cup B) = m^*\left(\bigcup D_n\right) \leq \sum m^*(D_n) = m^*(A) + m^*(B) + \sum_{n\geq 3} m^*(\emptyset) = 0 + m^*(B) + 0 = m^*(B) \;.$$

Since $m^*(B) \leq m^*(A\cup B)$ and $m^*(B) \geq m^*(A\cup B)$, $m^*(B) = m^*(A\cup B)$.

\subsection*{3 Measurable Sets and Lebesgue Measure}

{\bf Definition} A set $E$ is said to be {\bf (Lebesgue) measurable} if for each set $A\subset \mathbb{R}$ we have using (Lebesgue) outer measure, $m^*$, that $m^*(A) = m^*(A\cap E) + m^*(A \cap E^c)$. \\

Let $D_1 = A\cap E$, $D_2 = A\cap E^c$, $D_n = \emptyset$ for $n \geq 3$. Then $A = (A\cap E)\cup (A\cap E^c) = \bigcup D_n$. By Proposition 2,

$$m^*(A) = m^*\left((A\cap E)\cup (A\cap E^c)\right) = m^*\left(\bigcup D_n \right)  \leq \sum m^*(D_n) = m^*(A\cap E) + m^*(A\cap E^c) \;.$$

Since we always have $m^*(A) \leq m^*(A\cap E) + m^*(A\cap E^c)$, we see that $E$ is measurable if and only if $m^*(A) \geq m^*(A\cap E) + m^*(A\cap E^c)$ for each set $A$. Since the definition is symmetric, $E$ is measurable if and only if $E^c$ is measurable. For any $A$, $m^*(A\cap \emptyset) + m^*(A\cap \mathbb{R}) = m^*(\emptyset) + m^*(A) = m^*(A)$, which shows that both $\emptyset$ and $\mathbb{R}$ are measurable. \\

{\bf Lemma 6} If $m^*(E) = 0$, then $E$ is measurable.\\

Proof: Let $A\subset R$. Since $A\cap E \subset E$, $m^*(A\cap E) \leq m^*(E) = 0$ so $m^*(A\cap E) = 0$. Since $A\cap E^c \subset A$, $m^*(A\cap E^c) \leq m^*(A)$. Then $m^*(A\cap E) + m^*(A\cap E^c) = m^*(A\cap E^c) \leq m^*(A)$, which holds if and only if $E$ is measurable. \\


{\bf Lemma 7} If $E_1$ and $E_2$ are measurable, so is $E_1\cup E_2$.\\

Proof: Let $A\subset \mathbb{R}$. Since $E_2$ is measurable,

$$m^*(A\cap E_1^c) = m^*(A \cap E_1^c \cap E_2^c) + m^*(A \cap E_1^c \cap E_2) \;.$$ 

Also, by Corollary Stringbean,

\begin{align*}
m^*(A\cap (E_1 \cup E_2)) &= m^*\left[A\cap \left(E_1 \cup (E_2 \cap E_1^c)\right)\right]\\
&= m^*\left[(A\cap E_1)\bigcup (A\cap (E_2 \cap E_1^c)\right]\\
&\leq m^*\left(A\cap E_1\right) + m^*\left(A\cap (E_2 \cap E_1^c)\right)\\
&= m^*\left(A\cap E_1\right) + m^*\left(A\cap E_2 \cap E_1^c\right) \;.
\end{align*}

Therefore, 

\begin{align*}
m^*(A \cap (E_1\cup E_2)) + m^*(A\cap (E_1\cup E_2)^c) &= m^*(A \cap (E_1\cup E_2)) + m^*(A\cap E_1^c \cap E_2^c) \\
&\leq m^*\left(A\cap E_1\right) + m^*\left(A\cap E_2 \cap E_1^c\right) + m^*(A\cap E_1^c \cap E_2^c)\\
&= m^*\left(A\cap E_1\right) + m^*(A\cap E_1^c \cap E_2+ m^*(A\cap E_1^c \cap E_2^c)\\
&=m^*\left(A\cap E_1\right) + m^*(A \cap E_1^c)\\
&= m^*(A) \;.
\end{align*}

Since $A$ was arbitrary, this shows that $E_1 \cup E_2$ is measurable.\\

{\bf Corollary 8} The family $\mathfrak{M}$ of measurable sets is an algebra of sets. \\

If $A$ and $B$ are measurable sets, then $A\cup B$ is measurable by Lemma 7 so $A\cup B \in \mathfrak{M}$. From the definition of measurability, $A$ is measurable if and only if $A^c$ is measurable. So $A \in \mathfrak{M} \implies A^c \in \mathfrak{M}$. Therefore, $\mathfrak{M}$ is an algebra of sets. \\

{\bf Lemma 9} Let $A$ be any set and $E_1,..., E_n$ a finite sequence of disjoint meaurable sets. Then 

$$m^*\left(A\cap \left[\bigcup_{i=1}^n E_i\right]\right) =\sum_{i=1}^n m^*(A\cap E_i) \;. $$

Proof: Use induction on $n$. For $n = 1$, writing out each side of the expression immediately shows that equality holds. Assume the result holds for some $n \in \mathbb{N}$ and consider the case for $n+1$. Since $E_{n+1}$ is measurable, we have by the definition of measurability that for the set $A \cap \left[\bigcup_{i=1}^{n+1} E_i\right]$,

\begin{align*}
m^*\left(A\cap \left[\bigcup_{i=1}^{n+1} E_i\right]\right) &= m^*\left(A\cap \left[\bigcup_{i=1}^{n+1} E_i\right]\cap E_{n+1} \right) + m^*\left(A\cap \left[\bigcup_{i=1}^{n+1} E_i\right]\cap E_{n+1}^c \right)\\
&=m^*\left(A \cap E_{n+1} \right) + m^*\left(A\cap \left[\bigcup_{i=1}^{n+1} E_i\right]\cap E_{n+1}^c \right)\\
&\quad \quad (\text{because } E_{n+1} \cap E_i = \emptyset, i \neq n+1)\\
&=m^*\left(A \cap E_{n+1} \right) + m^*\left(A\cap \left[\bigcup_{i=1}^{n} E_i\right] \right)\\
&\quad \quad (\text{because } E_i \cap E_{n+1}^c = E_i, i\neq n+1, \text{ and } E_{n+1} \cap E_{n+1}^c = \emptyset )\\
&= m^*\left(A \cap E_{n+1} \right) +\sum_{i=1}^n m^*(A\cap E_i) \quad (\text{by inductive hypothesis})\\
&= \sum_{i=1}^{n+1} m^*(A\cap E_i) \;.
\end{align*}

Therefore the equality holds for all positive integers. \\

{\bf Lemma 9.5} If $C$ and $D$ are measurable sets then $C\cap D$ is measurable.\\

Proof: Note that for any set $A$, $A\cap (C \cap D) \subset A\cap C$ and $A \cap (C^c \cap D^c) \subset A \cap  C^c$, so $m^*(A\cap (C \cap D)) \leq m^*( A\cap C)$ and $m^*(A\cap (C^c\cap D^c)) \leq m^*( A\cap C^c)$.

\begin{align*}
m^*(A\cap (C \cap D))+ m^*(A\cap (C^c\cap D^c))
&\leq m^*(A \cap C)+ m^*( A\cap C^c)
= m^*(A)
\end{align*}

Since $m^*(A) \geq m^*(A\cap (C \cap D))+ m^*(A\cap (C^c\cap D^c))$, $C\cap D$ is measurable.\\

{\bf Theorem 10} The collection $\mathfrak{M}$ is a $\sigma$-algebra. Moreover, every set with outer measure zero is measurable.\\

Proof: We have already seen that $\mathbb{R}$ (and $\emptyset$) is measurable and that $\mathfrak{M}$ is an algebra of sets so that the collection is closed under complementation. We need to last check that if $E = \bigcup_{n \in \mathbb{N}} A_n$ with $A_n \in \mathfrak{M}$ for every $n$, that $E \in \mathfrak{M}$ as well. By taking $E_1 = A_1$, $E_n = A_n \backslash \left(\bigcup_{i=1}^{n-1} A_i\right)$ for $n \geq 2$ we have a pairwise disjoint countable sequence of sets $(E_n)_{n \in \mathbb{N}}$ such that $\bigcup E_n = \bigcup A_n = E$. Since each $A_n$ is measurable, repeated use of the fact that the union of measurable sets is measurable and the fact that the complement of a measurable set is measurable gives that $\left(\bigcup_{i=1}^{n-1} A_i \right)^c$ is measurable. By Lemma 9.5, each $E_n$ is then also measurable. So $E = \bigcup E_n$ is the countable union of disjoint measurable sets. Let $A$ be any set and for each $n$ define $F_n = \bigcup_{i = 1}^n E_i$. Repeated use of Lemma 7 shows that $F_n$ is measurable and since $F_n \subset E$, $F_n^c \supset E^c$ and hence $A \cap F_n^c \supset A \cap E^c$. Also, $F_n^c$ is measurable as $F_n$ is measurable.

\begin{align*}
m^*(A) &= m^*(A \cap F_n) + m^*(A \cap F_n^c) \\
&\geq  m^*(A \cap F_n) + m^*(A \cap E^c)\\
&= m^*\left(A \cap \left[\bigcup_{i=1}^n E_n\right]\right) + m^*(A\cap E^c) \\
&= \sum_{i=1}^n m^*(A\cap E_i) + m^*(A\cap E^c) \quad (\text{ by Lemma 9})\\
&\text{Then,}\\
&m^*(A) = \text{lim}_{n\rightarrow \infty} m^*(A) \geq \text{lim}_{n\rightarrow \infty} \left(\sum_{i=1}^n m^*(A\cap E_i) + m^*(A\cap E^c)\right)\\
& m^*(A) \geq \sum_{i=1}^\infty m^*(A\cap E_i) + m^*(A\cap E^c) \geq m^*(A \cap E) + m^*(A \cap E^c) \;.\\
\end{align*}


The last inequality follows from:
$$A \cap E = A \cap \left[\bigcup_{i=1}^\infty E_n \right] = \bigcup_{i=1}^\infty \left(A \cap E_i\right)$$
$$\implies m^*(A\cap E) = m^*\left( \bigcup_{i=1}^\infty \left(A \cap E_i\right)\right) \leq \sum_{i = 1}^\infty m^*(A \cap E_i) \quad \text{ (by Proposition 2)}\;.$$

Since $A\subset \mathbb{R}$ was arbitrary and we showed $m^*(A) \geq m^*(A \cap E) + m^*(A \cap E^c)$, conclude that $E$ is measurable. That is, if a set $E$ is the countable union of measurable sets, $E$ is measurable. Therefore, $\mathfrak{M}$ meets the three conditions necessary to conclude that $\mathfrak{M}$ is a $\sigma$-algebra. \\

{\bf Lemma 11} For any $a \in \mathbb{R}$, the interval $(a,\infty)$ is measurable. \\

Proof: Let $A\subset \mathbb{R}$ be arbitrary. Define $A_1 = A\cap (a,\infty)$ and $A_2 = A \cap (a, \infty)^c = A\cap (-\infty,a]$. As we have seen, it is the case for any $A$ that $m^*(A) \leq m^*(A_1) + m^*(A_2)$, so to show that $(a,\infty)$ is measurable we need to also show that $m^*(A) \geq m^*(A_1) + m^*(A_2)$. If $m^*(A) = \infty$, then $m^*(A) = \infty \geq  m^*(A_1) + m^*(A_2)$ regardless of the values of $m^*(A_1)\leq \infty$ and $m^*(A_2) \leq \infty$. Otherwise for $m^*(A) \in [0,\infty)$, given $\epsilon > 0$ there is a cover of $A$ by a collection open intervals $\{I_n\}$, such that 

$$ \sum l(I_n) < m^*(A) + \epsilon \;.$$

For each $n$, define $I_n' = I_n \cap (a,\infty)$ and $I_n'' = I_n \cap (-\infty, a]$. Then each $I_n',I_n''$ is either an open interval or the empty set. Also $l(I_n) = l(I_n') + l(I_n'') = m^*(I_n') + m^*(I_n'')$. Since $A \subset \bigcup I_n$, 
$$A_1 = A \cap (a,\infty) \subset \left(\bigcup I_n \right) \cap (a,\infty) = \bigcup (I_n \cap (a,\infty)) = \bigcup I_n' $$
$$A_1 \subset \bigcup I_n' \implies m^*(A_1) \leq m^*\left(\bigcup I_n'\right) \leq \sum m^*(I_n')$$
$$A_2 = A\cap (-\infty,a] \subset \left(\bigcup I_n \right) \cap (-\infty, a] = \bigcup (I_n \cap (-\infty,a]) = \bigcup I_n''$$
$$A_2 \subset \bigcup I_n'' \implies m^*(A_2) \leq m^*\left(\bigcup I_n''\right) \leq \sum m^*(I_n'') \;.$$

Therefore,
\begin{align*}
m^*(A_1) + m^*(A_2) &\leq \sum m^*(I_n') + \sum m^*(I_n'') \\
&= \sum \left(m^*(I_n') + m^*(I_n'')\right) \quad \text{ (see note below about convergence)}\\
&= \sum m^*(I_n)\\
&< m^*(A) + \epsilon \;.
\end{align*}

Since this inequality holds for every $\epsilon > 0$, conclude that $m^*(A_1) + m^*(A_2) \leq m^*(A)$, which shows that $(a,\infty)$ is measurable as $A$ was an arbitrary set. Note that we can combine the sums in the string of inequalities above since if one of these series did not converge to a finite value, the fact that $\sum m^*(I_n'), \sum m^*(I_n'') \leq \sum m^*(I_n) = \sum l(I_n) < m^*(A) + \epsilon$ would contradict the assumption that $m^*(A) < \infty$. \\

{\bf Theorem 12} Every Borel set is measurable. In particular each open set and each closed set is measurable. \\

Proof: Since the collection $\mathfrak{M}$ is a $\sigma$-algebra, each set of the form $(-\infty, a]$ is measurable as the complement of the measurable set $(a,\infty)$. Therefore, for each $n \in \mathbb{N}$, a set of the form $(-\infty, b-1/n]$ is measurable and so by condition $\Sigma_3$, $(-\infty, b) = \bigcup_{n} (-\infty, b-1/n]$ must also be measurable. Hence each open interval $(a,b)$ (with $a < b$) must be measurable as $(a,b) = (-\infty, b) \cap (a,\infty)$. By Proposition 2.8 (Proposition 8 of Chapter 2), every open set of real numbers is the union of a countable collection of disjoint open intervals. Since we just showed that open intervals are in $\mathfrak{M}$ and since $\mathfrak{M}$ is a $\sigma$-algebra, this means that every open set is the countable union of measurable sets and is thus also a measurable set. Therefore, $\mathfrak{M}$ is a $\sigma$-algebra that contains all the open sets. The Borel algebra $\mathfrak{B}$ is the $\sigma$-algebra generated by the open sets (the smallest $\sigma$-algebra containing the open sets. Thus $\mathfrak{B} \subset \mathfrak{M}$ and so any Borel set $B\in \mathfrak{B}$ is also a measurable set, $B \in \mathfrak{M}$. Any open set is a Borel set since $\mathfrak{B}$ contains all open sets. Since $\mathfrak{B}$ is a $\sigma$-algebra, any closed set is a Borel set as the complement of an open set. Therefore each open set of real numbers and each closed set of real numbers is measurable. \\

{\bf Definition} If $E$ is a measurable set, we define the Lebesgue measure of $E$, $m(E)$, as the outer measure of $E$. That is $m : \mathfrak{M} \rightarrow [0,\infty]$, $m(E) = m^*(E)$ is the set function obtained by restricting the set function $m^*$ to the family $\mathfrak{M}$ of measurable sets. \\

{\bf Proposition 13} Let $(E_i)$ be a sequence of measurable sets. Then,

$$m\left(\bigcup E_i\right) \leq \sum m(E_i) \;.$$

If the $E_i$ are pairwise disjoint,

$$m(\left(\bigcup E_i \right) = \sum m(E_i) \;.$$

Proof: If all $E_i$ are measurable, then $\bigcup E_i$ is measurable and so by the subadditivity of $m^*$ (Proposition 2),

$$m\left(\bigcup E_i\right) \leq m^*\left(\bigcup E_i\right) \leq \sum m^*(E_i) = \sum m(E_i) \;.$$

For the first $n$ terms of the sequence, we have in the finite case by Lemma 9 that,

\begin{align*}
m\left(\bigcup_{i=1}^n E_i \right) &=  m\left( \mathbb{R} \cap \left[ \bigcup_{i=1}^n E_i \right]\right) \\
&= m^*\left(\mathbb{R} \cap \left[\bigcup_{i=1}^n E_i \right]\right) \quad (\text{measurability of } \mathbb{R}, \bigcup_{i=1}^n E_i \text{ and closure under intersection})\\
&=\sum_{i=1}^n m^*(\mathbb{R}\cap E_i) \quad \text{(Lemma 9)}\\
&=\sum_{i=1}^n m^*(E_i)\\
\implies m^*\left(\bigcup_{i=1}^\infty\right) &\geq m^*\left(\bigcup_{i=1}^n E_i\right) = \sum_{i=1}^n m^*(E_i) \quad (\text{ because }\bigcup_{i=1}^\infty E_i \supset \bigcup_{i=1}^n E_i \;)\\
\implies m^*\left(\bigcup_{i=1}^\infty\right) &= \text{lim}_{n\rightarrow \infty} \; m^*\left(\bigcup_{i=1}^\infty\right) \geq \text{lim}_{n\rightarrow \infty} \; \sum_{i=1}^n m^*(E_i) = \sum_{i=1}^\infty m^*(E_i)
\end{align*}

This shows that $m^*\left(\bigcup_{i=1}^\infty\right) \geq  \sum_{i=1}^\infty m^*(E_i)$. We showed in the beginning of the proof that subadditivity, $m^*\left(\bigcup_{i=1}^\infty\right) \geq  \sum_{i=1}^\infty m^*(E_i)$, holds for any sequence $(E_i)$ of measurable sets so we can conclude that if the $E_i$ are pairwise disjoint, $m^*\left(\bigcup_{i=1}^\infty\right) =  \sum_{i=1}^\infty m^*(E_i)$.\\

{\bf Proposition 14} Let $(E_i)$ be a sequence of decreasing measurable sets, that is, a sequence with $E_{n+1} \subset E_n$ for each $n \in \mathbb{N}$. Let $m(E_1)<\infty$. Then,

$$m\left(\bigcap_{i=1}^\infty E_i \right) = \text{lim}_{n\rightarrow \infty} \; m(E_n) \;.$$

Proof: Let $E = \bigcap_{i=1}^\infty E_i$ and $F_i = E_i \backslash E_{i+1}$. Note that since the sequence is decreasing that $E_i \backslash E_{i+1} = E_i \cap (\bigcap_{k=i+1}^\infty E_k^c )$. Then $\bigcup_{i=1}^\infty F_i = E_1 \backslash E$. If $a \in \bigcup_{i=1}^\infty F_i$, $a \in F_i$ for some $i$. Then $a \in E_i$ but $a \not\in E_{i+1}$. Since $E_i \subset... \subset E_1$, $a \in E_1$ but $a \not\in E_{i+1}$. So $a \not\in \bigcap_{i=1}^\infty E_i$. Thus $a \in E_1 \backslash E$. If $a \in E_1 \backslash E$, then $a \in E_1$ and for some $i$, $a \not \in E_i$. The set $\{k : a \not\in E_k\}$ must have a minimal element $i \geq 2$, so we may assume that $a \not\in E_i$ but that $a \in E_n$ for $n < i$. Then $a \in E_{i-1} \backslash E_{i} = F_{i-1}$. Therefore $a \in \bigcup_{i=1}^\infty F_i$. The $F_i$ are pairwise disjoint since if $i\neq j$, we may assume without loss of generality that $i<j$. Then since $E_j \subset ... \subset E_{i+1}$ (if $j = i+1$, we have $F_i \cap F_j = E_i \cap E_j^c \cap E_j \cap E_{j+1}^c = \emptyset$),

$$F_i \cap F_j = (E_i \cap E_{i+1}^c ) \cap (E_j \cap E_{j+1}^c) = (E_i \cap E_{i+1}^c \cap ... \cap E_j^c ) \cap (E_j \cap E_{j+1}^c) = \emptyset \;.$$

By Proposition 13,
$$m(E_1 \backslash E) = \sum_{i=1}^\infty m(F_i) = \sum_{i=1}^\infty m(E_i \backslash E_{i+1}) \;.$$

Using Proposition 13 with $E_1 := E_1$, $E_2 := E$, $E_n := \emptyset$ for $n\geq 3$, we have that $m(E_1) = m(E\cup (E_1 \backslash E)) = m(E) + m(E_1 \backslash E)$. By similar reasoning, $m(E_i) = m(E_{i+1}) + m(E_i \backslash E_{i+1})$. Since $m(E_i) \leq m(E_1) < \infty$, we can subtract through to get $m(E_1\backslash E) = m(E_1) - m(E)$ and $m(E_i \backslash E_{i+1}) = m(E_i) - m(E_{i+1})$. As an alternative way to prove this, not first that since $\mathfrak{M}$ is a $\sigma$-algebra, the countable intersection of measurable sets $E$ is also measurable. Using the definition of measurability of $E$ and $E_{i+1}$ and the assumption that $m(E_i) \leq m(E_1) < \infty$.

\begin{align*}
m^*(E_1) &= m^*(E_1 \cap E) + m^*(E_1 \cap E^c)\\
m(E_1) &= m(E_1 \cap E) + m(E_1 \cap E^c) \quad (m^* = m \text{ for measurable sets})\\
&= m(E) + m(E_1 \cap E^c) \quad (E \subset E_1)\\
&= m(E) + m(E_1 \backslash E) \\
&\implies m(E_1 \backslash E) = m(E_1) - m(E) \;.\\
&\\
m^*(E_i) &= m^*(E_i \cap E_{i+1}) + m^*(E_i \cap E_{i+1})\quad (m^* = m \text{ for measurable sets})\\
m(E_i) &= m(E_i \cap E_{i+1}) + m(E_i \cap E_{i+1})\\
&= m(E_{i+1}) + m(E_i \backslash E_{i+1}) \quad (E_{i+1} \subset E_i)\\
&\implies m(E_i \backslash E_{i+1}) = m(E_i) - m(E_{i+1}) \;. 
\end{align*}

Thus,

\begin{align*}
m(E_1) -m(E) &= m(E_1 \backslash E)\\
&= \sum_{i=1}^\infty m(F_i)\\
&= \sum_{i=1}^\infty m(E_i \backslash E_{i+1})\\
&= \sum_{i=1}^\infty m(E_i) - m(E_{i+1}) \\
&= \text{lim}_{n\rightarrow \infty} \sum_{i=1}^n m(E_i) - m(E_{i+1})\\
&= \text{lim}_{n\rightarrow \infty} \left(m(E_1) - m(E_2) + m(E_2) - m(E_3) + ... + m(E_n) - m(E_{n+1})\right)\\
&= \text{lim}_{n\rightarrow \infty} \left( m(E_1) - m(E_{n+1})\right)\\
&= m(E_1) - \text{lim}_{n\rightarrow \infty} m(E_{n+1})\\
&= m(E_1) - \text{lim}_{n\rightarrow \infty} m(E_n) \;.
\end{align*}

Since $m(E_n) \leq m(E_1) < \infty$ for each $n$, we can rearrange using addition to get $m\left(\bigcap_{n=1}^\infty E_n\right) = m(E) = \text{lim}_{n\rightarrow \infty} m(E_n)$. \\


{\bf Problem 9} Show that if $E$ is a measurable set the each translate $E + y$, $y \in \mathbb{R}$, of $E$ is also measurable.\\

{\bf Lemma 9A} Let $D \subset \mathbb{R}$ and $x \in \mathbb{R}$. Then $(D+x)^c = D^c + x$.\\

Proof: Let $a \in (D+x)^c$. Then $a \not\in D+x$ and so $a \neq d + x$ for any $d \in D$. This means that $a - x \neq d$ for any $d \in D$ and so $a-x \not\in D$. Thus $a - x \in D^c$ and so $a \in D^c + x$. \\

Let $a \in D^c + x$. Then $a = \tilde{d} + x$ for some $\tilde{d} \in D^c$. Thus $a - x = \tilde{d} \in D^c$. Then $a-x \neq d$ for any $d \in D$ which means also $a \neq d+x$ for any $d \in D$. Then it cannot be the case that $a \in D+x$, so $a \in (D+x)^c$. \\

{\bf Lemma 9B} Let $C,D \subset \mathbb{R}$ and $x \in \mathbb{R}$. Then $(C\cap D) -x = (C-x)\cap (D-x)$.\\

Proof: Let $a \in (C\cap D) - x$. Then $a = b-x$ for some $b \in C\cap D$ and so $a = b-x$ for some $b$ such that $b \in C$ and $b \in D$. Thus $a=b-x \in C-x$ and $a = b-x \in D-x$. Therefore, $a \in (C-x) \cap (D-x)$. \\

Let $a \in (C-x)\cap (D-x)$. Then $a \in C-x$ and $a \in D-x$. Therefore, $a = c-x$ for some $c \in C$ and $a = d-x$ for some $d \in D$. But since $c-x = d-x$, $c=d$ and so $c=d \in C\cap D$. Thus $a = c-x$ for some $c \in C\cap D$ and therefore $a \in (C\cap D) -x$. \\

Now let $E$ be a measurable set, $A \subset \mathbb{R}$, and $y \in \mathbb{R}$.\\

\begin{align*}
m^*[A \cap (E+y)] + m^*[A \cap (E+y)^c] &= m^*[A\cap (E+y)] + m^*[A \cap (E^c+y)] \quad \text{(Lemma 9A)}\\
&= m^*[\left(A\cap (E+y)\right) - y] + m^*[\left(A \cap (E^c +y\right) - y] \quad \text{(Problem 7)}\\
&=m^*[(A-y)\cap(E+y-y)] + m^*[(A-y)\cap(E^c+y-y)] \quad \text{(Lemma 9B)}\\
&=m^*[(A-y)\cap(E+0)] + m^*[(A-y)\cap(E^c+0)]\\
&= m^*[(A-y)\cap E] + m^*[(A-y)\cap E^c]\\
&= m^*(A-y)\quad \text{(Since } E \text{ is measurable)}\\
&= m^*(A) \quad \text{(Problem 7)}\;.
\end{align*}

Since $A$ and $y$ were arbitrary, we have shown that for any translate $E+y$ of $E$, that $m^*(A) = m^*(A \cap (E+y)) + m^*(A \cap (E+y)^c)$ for any set $A$. Therefore, $E+y$ is measurable.\\

{\bf Alternative Proof}\\
Let $\{I_n\}$ be any cover of $E$ by open intervals. Then $\{I_n + y\}$ is a cover of $(E + y)$ by open intervals since if $x \in E+y$ then $x - y \in E$ so $x-y \in I_n$ for some interval in $\{I_n\}$. Thus $x \in I_n + y$. Also, for any interval $I_n = (a_n,b_n)$, $l(I_n) = b_n - a_n = (b_n +y ) - (a_n+y) = l(I_n + y)$. Since $A\cap E \subset E$ and $A\cap (E+y) \subset E+y$, any cover $\{I_n\}$ of $E$ by open intervals will be a cover of $A\cap E$ and since the corresponding cover $\{I_n + y\}$ contains $E+y$, $A\cap (E+y) \subset \bigcup (I_n+y)$. This means that the values of the sums in the set $\{\sum l(I_n) : (A\cap E) \subset \bigcup I_n\}$ are the same as the values of the sums in the set $\{\sum l(I_n + y) : (A\cap (E+y)) \subset \bigcup (I_n + y)\}$. Therefore, $m^*(A \cap E) = m^*(A\cap (E+y))$. Similarly, $m^*(A\cap E^c) = m^*(A \cap (E+y)^c) = m^*(A \cap (E^c + y))$ (Lemma 9A). By Problem 7, $m^*$ is translation invariant and since $E$ is measurable we have:

$$m^*(A\cap (E+y)) + m^*(A \cap (E+y)^c) = m^*(A\cap E) + m^*(A \cap E^c) = m^*(A) \;.$$

Since $A$ and $y$ were arbitrary this shows that if $E$ is measurable, $E+y$ is measurable. \\


{\bf Problem 10}\\

Show that if $E_1$ and $E_2$ are measurable, then 

$$m(E_1\cup E_2) + m(E_1\cap E_2) = m(E_1) + m(E_2) \;.$$

First if $m(E_1) = \infty$ or $m(E_2) = \infty$, then $E_1,E_2 \subset E_1\cup E_2$ implies that $m(E_1\cup E_2) = \infty$ as well. In this case the equality holds since both sides are $\infty$. Otherwise assume that both $m(E_1)$ and $m(E_2)$ are finite.\\

Since $E_1$ and $E_2$ are measurable and $\mathfrak{M}$ is a $\sigma$-algebra, the sets $E_1\cup E_2$, $E_1 \cap E_2^c$, $E_2\cap E_1^c$, and  $E_1\cap E_2$ are all also measurable. Then for each set mentioned, $m^* = m$. Since $m(E_1),m(E_2) < \infty$, each of these sets mentioned also has finite measure so that addition and subtraction are meaningful in the equations below (no expressions like $\infty - \infty$ arise). All sets mentioned are contained in either $E_1$ or $E_2$ or both except $E_1 \cup E_2$. If $m(E_1 \cup E_2) = \infty$, then for every cover $\{I_n\}$ of $E_1\cup E_2$ by open intervals, $\sum l(I_n) = \infty$. However, since there exists a collection $\{I_j\}$ and a collection $\{I_k\}$ such that $E_1 \subset \bigcup I_j$, $E_2 \subset \bigcup I_k$ and $\sum l(I_j) < \infty$, $\sum l(I_k) < \infty$, we have that $\{I_n\} = \{I_k\} \cup \{I_j\}$ is a collection of open intervals such that $E_1\cup E_2 \subset \bigcup I_n$ and $\sum l(I_n) = \sum I_j + \sum I_k < \infty$. This is a contradiction, so $m(E_1 \cup E_2) < \infty$. Also by Lemma 9, setting $A = \mathbb{R}$, we have that if $C_1,...,C_n$ are disjoint measurable sets then $m(C_1\cup ... \cup C_n) = m^*(C_1 \cup ... \cup C_n) = \sum_{i=1}^n m^*(C_i) = \sum_{i=1}^n m(C_i)$. That is, the Lebesgue measure of the union of disjoint measurable is equal to the sum of the Lebesgue measures of the sets. We use these facts to decompose the measure of sets into the sums of measures of smaller components and then manipulate the results to arrive at the identity.

$$(1) \quad m(E_1\cup E_2) = m(E_1 \cap E_2^c) + m(E_2 \cap E_1^c) + m(E_1 \cap E_2)$$
$$(2) \quad m(E_1) = m(E_1\cap E_2^c) + m(E_1\cap E_2)$$
$$(3) \quad m(E_2) = m(E_2 \cap E_1^c) +m(E_1 \cap E_2)$$
$$\text{ Add equations (2) and (3) to get (4) }$$
$$(4) \quad m(E_1) + m(E_2) = m(E_1\cap E_2^c) + m(E_2 \cap E_1^c) + 2m(E_1\cap E_2)$$
$$(5) \quad m(E_1) + m(E_2) - m(E_1\cap E_2) = m(E_1\cap E_2^c) + m(E_2 \cap E_1^c) + m(E_1\cap E_2)$$
$$\text{ Compare equation (5) to equation (1) }$$
$$(6) \quad m(E_1 \cup E_2) = m(E_1) + m(E_2) - m(E_1\cap E_2)$$
$$(7) \quad m(E_1 \cup E_2) + m(E_1\cap E_2)= m(E_1) + m(E_2) \;.$$
 
{\bf Problem 12}\\

Let $(E_i)$ be a sequence of disjoint measurable sets and $A \subset \mathbb{R}$.

$$m^*\left(A \cap \bigcup_{i=1}^\infty E_i \right) = \sum_{i=1}^\infty m^*(A \cap E_i) \;.$$

Proof:
$$
m^*\left(A \cap \bigcup_{i=1}^\infty E_i \right) = m^*\left(\bigcup_{i=1}^\infty (A \cap E_i) \right)
\leq \sum_{i = 1}^\infty m^*(A \cap E_i) \quad \text{(by Proposition 2)}\; .
$$

Since $\bigcup_{i=1}^\infty E_i\supset \bigcup_{i=1}^n E_i$ for every $n$, $A\cap \bigcup_{i=1}^\infty \supset A \cap \bigcup_{i=1}^n E_i$ for every $n$.
$$
      m^*\left(A\cap \bigcup_{i=1}^\infty E_i\right)
      \ge m^*\left(A\cap \bigcup_{i=1}^n E_i\right)
      = \sum_{i=1}^n m^*(A\cap E_i),
$$
    where the equality comes from Lemma 9. Since the left hand side is 
    independent of $n$, we have
    $$
      m^*\left(A\cap \bigcup_{i=1}^\infty E_i\right) = \text{lim}_{n\rightarrow \infty} m^*\left(A\cap \bigcup_{i=1}^\infty E_i\right)  \geq \text{lim}_{n\rightarrow \infty}
      \sum_{i=1}^n m^*(A\cap E_i) = \sum_{i=1}^\infty m^*(A\cap E_i)
    $$
    
    This means $m^*\left(A\cap \bigcup_{i=1}^\infty E_i\right) \geq \sum_{i=1}^\infty m^*(A\cap E_i)$ as well.\\
    
{\bf Proposition 15} Let $E$ be a given set. The following five statements are equivalent.\\

i. $E$ is measurable.\\
ii. Given $\epsilon > 0$ there is an open set $O \supset E$ such that $m^*(O \backslash E) < \epsilon$. \\
iii. Given $\epsilon > 0$ there is a closed set $F \subset E$ such that $m^*(E \backslash F) < \epsilon$.\\
iv. There is a $G \in G_{\delta}$ with $E \subset O$ such that $m^*(G \backslash E) = 0$.\\
v. There is an $F \in F_\sigma$ with $F \subset E$ such that $m^*(E \backslash F) = 0$. \\

If $m^*(E) < \infty$, the above statements are equivalent to:\\

vi. Given $\epsilon > 0$, there is a finite union $U$ of open intervals such that $m^*(U \bigtriangleup E) < \epsilon$.\\

{\bf Problem 13} Prove Proposition 15.\\

(a) Assume for part (a) that $m^*(E) < \infty$.\\

(i) $\implies$ (ii)\\
Suppose $E$ is measurable and let $\epsilon > 0$. By Proposition 5, since $E \subset \mathbb{R}$, there is an open set $O \supset E$ such that $m^*(O) \leq m^*(E) + \epsilon$. As we saw in the proof of this proposition, the inequality can be made strict if $m^*(E) < \infty$ (or just start with $m^*(O) \leq m^*(E) + \epsilon / 2$). Then since $E$ is measurable,

\begin{align*}
m^*(O) &= m^*(O\cap E)+ m^*(O\cap E^c)\\
&= m^*(E) + m^*(O\backslash E) \\
\implies m^*(O\backslash E) &= m^*(O) - m^*(E) \\
&< m^*(E) + \epsilon - m^*(E) \\
&= \epsilon \;.
\end{align*}

However, there appears to be no reason to use the assumption that $E$ is measurable. Since $E\subset O$, we can write $O$ as a disjoint union $O = (O\backslash E) \cupdot E$ so that $m^*(O) < m^*(E) + \epsilon \implies m^*(O\backslash E) = m^*(O) - m^*(E) < \epsilon$. Note that $m^*(O) < \infty$ as $m^*(O) < m^*(E) + \epsilon < \infty$. \\

(ii) $\implies$ (vi)\\
Assume (ii) holds and let $\epsilon > 0$. There exists an open set $O\supset E$ such that $m^*(O \backslash E) < \epsilon /2$. By Proposition 8, there is a disjoint collection of open intervals $\{I_n\}$ such that $O = \bigcupdot_{n=1}^\infty I_n$ (If the collection $\{I_n\}$ is actually finite, with say $\bigcupdot_{n=1}^p I_n = O$, set $I_n = \emptyset$ for $n > p$). Since $O\supset E$, $O = (O\backslash E) \bigcupdot E$ is a disjoint union and $m^*(O) = m^*(O \backslash E) + m^*(E)< \epsilon + m^*(E)$. This implies that $m^*(O) < \infty$.

$$\infty > m^*\left(\bigcupdot I_n\right) = \sum_{n=1}^\infty m^*(I_n) \;.$$
Since the series converges to a finite value, the nonnegative sequence of partial sums converges to 0. This implies that there is an $N$ such that $\sum_{n=N}^\infty m^*(I_n) < \epsilon /2$. Let $U = \bigcupdot_{n=1}^N I_n$. From $E \subset O$ and $U \subset O$ it follows that

$$m^*(E\backslash U) \leq m^*(O \backslash U) = m^*(O)-m^*(U) = \sum_{n=1}^\infty m^*(I_n) - \sum_{n=1}^N m^*(I_n) = \sum_{n=N}^\infty m^*(I_n) < \epsilon/2 \;.$$

Since $U \subset O$, $U\backslash E \subset O \backslash E$ and $m^*(U\backslash E) \leq m^*(O \backslash E) < \epsilon/2$. For the finite union of open intervals $U$, 

$$m^*\left( U \bigtriangleup E\right) = m^*\left( (U \backslash E)\cup (E \backslash U)\right) \leq m^*(U \backslash E) + m^*(E \backslash U) < \epsilon/2 + \epsilon /2 = \epsilon \;.$$
$$ $$

(vi) $\implies$ (ii)\\
Assume (vi) holds and let $\epsilon > 0$. By Proposition 5, since $E$ is a set of real numbers there is an open set $O$ such that $m^*(O) < m^*(E) + \epsilon < \infty$ (again the inequality can always be made strict so long as $m^*(E) < \infty$). Then $m^*(O\backslash E) = m^*(O) - m^*(E) < \epsilon$. There appears to be no need to use (vi). If we want to explicitly involve (vi), consider that we can show (iv) $\implies$ (i) and use (i) $\implies$ (ii) to conclude that (iv) $\implies$ (i). The strategy used in this case is then better described as (i)$\implies$(ii)$\implies$(vi)$\implies$(i).\\

Assume (iv) holds and let $A \subset \mathbb{R}$ and $\epsilon > 0$. As we have seen, it is always the case that $m^*(A) \leq m^*(A \cap E) + m^*(A \cap E^c)$. There is a finite union of open intervals $U$ such that $m^*(U \bigtriangleup E)  = m^*\left[(E\cap U^c)\cup (U \cap E^c)\right] < \epsilon / 2$. As $U$ is the union of open sets, $U$ is open. By Theorem 12, $U$ is measurable.

\begin{align*}
m^*(A\cap E) + m^*(A \cap E^c) &\\
= [m^*(A\cap E \cap U) + m^*(A \cap E \cap U^c)]
+ [m^*(A\cap E^c \cap U) + m^*(A \cap E^c \cap U^c)]& \\
 \leq [m^*(A \cap U) + m^*(A \cap E \cap U^c)] + [m^*(A\cap E^c \cap U) + m^*(A \cap E^c \cap U^c)]&\\
 \leq [m^*(A \cap U) + m^*(E \cap U^c)] + [m^*(A\cap E^c \cap U) + m^*(A \cap E^c \cap U^c)]&\\
 \leq [m^*(A \cap U) + m^*(E \cap U^c)] + [m^*(E^c \cap U) + m^*(A \cap E^c \cap U^c)]&\\
 \leq [m^*(A \cap U) + m^*(E \cap U^c)] + [m^*(E^c \cap U) + m^*(A \cap U^c)] &\\
= [m^*(A \cap U) + m^*(A \cap U^c)] + [m^*(E^c \cap U) + m^*(E \cap U^c)]& \\
 = m^*(A) + m^*(E^c \cap U) + m^*(E \cap U^c) &\\
= m^*(A) + m^*(U\cap E^c) + m^*(E \cap U^c)&\\
< m^*(A) + \epsilon & \\
\implies m^*(A\cap E) + m^*(A \cap E^c) \leq m^*(A) \quad \text{as } \epsilon \text{ was arbitrary}. &
\end{align*}

The last inequality above follows from:

$$m^*(E \cap U^c) \leq m^*\left[(E\cap U^c)\cup (U \cap E^c)\right] < \epsilon /2 $$ 
$$m^*(U \cap E^c) \leq  m^*\left[(E\cap U^c)\cup (U \cap E^c)\right] < \epsilon /2 $$ 
$$m^*(E \cap U^c) + m^*(U\cap E^c) < \epsilon \;. $$

Therefore, (iv) implies that $E$ is measurable. This (i) and so (ii) follows. \\

(b) Let $E$ be a given set.\\

(i) $\implies$ (ii)\\
Suppose $E$ is measurable. By part (a), if $m^*(E) < \infty$ then (i) $\implies$ (ii). If instead $m^*(E) = \infty$, then $E$ cannot be bounded. If $E \subset [-M,M]$ for some $M \in \mathbb{R}$ then the contradiction $\infty = m^*(E) \leq 2M$ follows. Define $E_k = \{e \in E : |e| \leq k\}$ for each $k \in \mathbb{N}$. Then $(E_k)$ is an increasing sequence of sets. Each $E_k$ is a bounded set of real numbers (or possibly empty) with $m^*(E_k) < \infty$. For each $E_k$ there is by Proposition 5 an open set $O_k \supset E_k$ with $m^*(O_k) < m^*(E_k) + \epsilon/2^k < \infty \implies m(O_k \backslash E_k) < \epsilon /2^k$. Set $O = \bigcup_{k=1}^\infty O_k$. Since $E\subset \mathbb{R}$, for any element $e \in E$ there is a $k$ such that $|e| \leq k$ and $e \in E_n\subset O_n \subset O$ for all $n \geq k$. So $E_k \uparrow E$ and $E \subset O$. 

\begin{align*}
m^*(O \backslash E) &= m^*\left((\bigcup_{k=1}^\infty O_k) \cap E)\right) \\
&= m^*\left(\bigcup_{k=1}^\infty (O_k \cap E) \right)\\
&\leq \sum_{k=1}^\infty m^*(O_k \cap E) \\
&\leq \sum_{k=1}^\infty m^*(O_k \cap E_k) \quad (E_k \subset E \implies O_k \cap E \subset O_k \cap E_k)\\
&< \sum_{k=1}^\infty \epsilon 2^{-k} \\
& = \epsilon \; .
\end{align*}

Conclude that $O$ is an open set with $O \supset E$ and $m^*(O \cap E) < \epsilon$. \\

(ii) $\implies$ (iv)\\
For any $k \in \mathbb{N}$, there is an open set $O_k \supset E$ such that $m^*(O_k \backslash E) < 1/k$. Let $G = \cap_{k=1}^\infty O_k$. For any $N \in \mathbb{N}$,

\begin{align*}
m^*(G \backslash E) &= m^*\left(\left(\bigcap_{k=1}^\infty O_k \right)\cap E\right)\\
&\leq m^*\left(\left(\bigcap_{k=1}^N O_k \right)\cap E\right) \quad (\cap_{k=1}^\infty O_k \subset \cap_{k=1}^N O_k)\\
&\leq m^*(O_N \backslash E) \quad (\cap_{k=1}^N O_k \subset O_N)\\
&< 1/N\\
\implies m^*(G \backslash E) &= \text{lim}_{N \rightarrow \infty} m^*(G\backslash E) \leq \text{lim}_{N \rightarrow \infty} 1/N = 0 \;.
\end{align*}

Since $m^*(G\backslash E) \geq 0$ by definition of $m^*$, conclude that for the countable intersection of open sets $G \in G_\delta$, $m^*(G \backslash E) = 0$.\\

(iv) $\implies$ (i)\\
Let $A \subset \mathbb{R}$ and by (iv) let $G \in G_\delta$ such that $m^*(G \backslash E) = 0$. Since $\mathfrak{M}$ is a $\sigma$-algebra by Theorem 10, $G = \bigcap_{k=1}^\infty G_k = \left(\bigcup_{k=1}^\infty G_k^c \right)^c \in \mathfrak{M}$ using $\sigma$-algebra properties and the fact that each open set $G_k \in \mathfrak{M}$. As noted several times in this section, $m^*(A) \leq m^*(A \cap E) + m^*(A \cap E^c)$ whether or not $E$ is measurable. For the reverse inequality,

\begin{align*}
m^*(A \cap E) + m^*(A \cap E^c) &= m^*(A\cap E) + m^*(A\cap E^c) \cap G) + m^*((A\cap E^c) \cap G^c) \\
&= m^*(A \cap E) + m^*(A \cap (E^c \cap G)) + m^*(A \cap (E^c \cap G^c))\\
&= m^*(A \cap E) + m^*(A \cap (G\backslash E)) + m^*(A\cap G^c) \quad (E\subset G \implies G^c \subset E^c)\\
&\leq m^*(A \cap E) + m^*(G\backslash E) + m^*(A\cap G^c) \quad (A\cap (G\backslash E) \subset G\backslash E) \\
& = m^*(A \cap E) + 0 + m^*(A\cap G^c) \\
& \leq m^*(A \cap G) + m^*(A\cap G^c) \quad (A\cap E \subset A \cap G)\\
&=m^*(A) \quad (G \text{ is measurable}) \;.
\end{align*}

Therefore $m^*(A\cap E) + m^*(A \cap E^c) \leq m^*(A)$, from which we conclude that $E$ is measurable. \\

(c) Let $E$ be a given set.\\

(i) $\implies$ (iii)\\
Assume that $E$ is measurable. Then $E^c$ is also measurable. By part (b), if (i) holds then (ii) holds as well so there is an open set $O \supset E^c$ such that $m^*(O \backslash E^c) < \epsilon$. But this means $\epsilon > m^*(O \cap (E^c)^c) = m^*(O\cap E)$. Since $E^c \subset O$, $O^c \subset E$. This gives a closed set, $O^c$, such that $O^c \subset E$ and $m^*(E \backslash O^c) = m^*(E \cap (O^c)^c) = m^*(E \cap O) < \epsilon$. \\

(iii) $\implies$ (v)\\
For each $n \in \mathbb{N}$, there is by (iii) a closed set $F_n\subset E$ such that $m^*(E \backslash F_n) < 1/n$. Let $F = \bigcup_{n=1}^\infty F_n$. Then $F \in F_\sigma$ is also a closed set and for any $N \in \mathbb{N}$,

\begin{align*}
m^*(E \backslash F) = m^*\left[E \cap \left(\bigcup_{n=1}^\infty F_n\right)^c\right]
&= m^*\left[E \cap \left(\bigcap_{n=1}^\infty F_n^c\right)\right]\\
&= m^*\left[\bigcap_{n=1}^\infty (E\cap F_n^c)\right]\\
&\leq m^*\left[\bigcap_{n=1}^N (E\cap F_n^c)\right]\quad \left(\bigcap_{n=1}^\infty F_n^c\subset \bigcap_{n=1}^N F_n^c\right) \\
&\leq m^*(E\cap F_n^c)\quad \left(\bigcap_{n=1}^N F_n^c\subset F_N^c\right)\\
&= m^*(E \backslash F_N) < 1/N \implies m^*(E \backslash F) \leq 0 \;.
\end{align*}

Since $m^*(E\backslash F) \geq 0$ by definition of $m^*$, conclude that $F \in F_\sigma$ satisfies the conditions of (v). \\

(v) $\implies$ (i)\\
Let $F \in F_\sigma$ such that $F \subset E$ and $m^*(E\backslash F) = 0$. As the union of closed (and therefore measurable) sets, $F$ is measurable. Let $A \subset \mathbb{R}$. Then $m^*(A) \leq m^*(A\cap E) + m^*(A\cap E^c)$. To show the reverse inequality,

\begin{align*}
m^*(A\cap E) + m^*(A\cap E^c) &= m^*((A\cap E)\cap F) + m^*((A\cap E)\cap F^c) + m^*(A\cap E^c)\\
&= m^*(A\cap(E\cap F)) + m^*(A\cap (E \backslash F)) + m^*(A \cap E^c) \\
&= m^*(A\cap F) + m^*(A\cap (E \backslash F)) + m^*(A \cap E^c) \\
&\leq m^*(A\cap F) + m^*(E \backslash F) + m^*(A \cap E^c) \\
&= m^*(A\cap F) +0 + m^*(A \cap E^c) \\
&\leq m^*(A \cap F) + m^*(A \cap F^c)\\
&= m^*(A)
\end{align*}

Therefore, $m^*(A) = m^*(A \cap E) + m^*(A \cap E^c)$, which shows that $E$ is measurable. \\

{\bf Definition} A {\bf ternary expansion} of $x \in [0,1]$ is a sequence $(a_n)$ with $0\leq a_n < 3$ such that 
$$x = \sum_{n=1}^\infty \frac{a_n}{3^n} \;.$$

{\bf Definition} The {\bf Cantor Ternary Set} $C$ consists of all those real numbers $[0,1]$ that have a ternary expansion $(a_n)$ for which $a_n$ is never 1 (if $x$ has two ternary expansions, we put $x$ in $C$ if one of the expansions has no term equal to 1). The set $C$ is closed and obtained by first removing $(1/3, 2/3)$ from $[0,1]$, then removing $(1/9,2/9)$ from $[0,1/3]$ and $(7/9,8/9)$ from $[2/3,1]$, and so on. Using the definition of ternary expansion,

$$C =  \Bigg\{ \sum_{n=1}^\infty \frac{a_n}{3^n} : a_n \in \{0, 1, 2\} \Bigg\}$$

{\bf Problem 14}\\

(a) Prove that the Cantor Ternary Set has measure 0. \\

Define,
\begin{align*}
E_1 &= [0/3,1/3] \cup [2/3, 3/3]\\
E_2 &= [0/9,1/9]\cup[2/9,3/9]\cup [6/9,7/9]\cup[8/9,9/9]\\
E_3 &= [0/27,1/27]\cup [2/27,3/27] \cup [6/27, 7/27]\cup[8/27,9/27] \cup [18/27, 19/27]\cup [20/27,21/27]\\
&\quad \quad \cup [24/27,25/27]\cup [26/27, 27/27]\\
\vdots
\end{align*}

Continue such that $E_{n+1}$ is obtained by removing the open interval making up the middle third of each closed interval in the union forming $E_n$. Then $E_{n+1} \subset E_n$ for all $n$ and each $E_n$ is measurable as a union of closed intervals (which are measurable by Theorem 12). For each $n$, $E_n$ is the union of $2^n$ closed intervals, each of length $(1/3)^n$. For each $n \geq 2$, $E_n$ is obtained by removing $2^{n-1}$ open intervals each of length $(1/3)^n$ from $E_{n-1}$.
\begin{align*}
m^*(E_n) = m(E_n) &= m([0,1]) -\sum_{k=1}^n \frac{2^{k-1}}{3^k}\\
&= 1 -\sum_{k=1}^n \frac{2^{k-1}}{3^k}\\
&= 1- \sum_{k=1}^n \frac{1}{2}\left(\frac{2}{3}\right)^k \\
&= 1 - \left(1-\left(\frac{2}{3}\right)^n\right)\\
&= \left(\frac{2}{3}\right)^n \;.
\end{align*}

The sequence of measurable sets $(E_n)$ is decreasing with $m(E_1) = 2/3 < \infty$. By Proposition 14:

$$m(C) = m\left(\bigcap_{i=1}^\infty E_i\right) = \text{lim}_{n\rightarrow \infty} m(E_n) = \text{lim}_{n\rightarrow \infty} \left(\frac{2}{3}\right)^n = 0 \;.$$

\subsection*{4 A Nonmeasurable Set}

{\bf Lemma 16} Let $E \subset [0,1)$ be a measurable set. Then for each $y \in [0,1)$ the set $E \dot{+} y$ is measurable and $m(E\dot{+}y) = m(E)$.\\

Proof: Let $E_1 = E\cap[0,1-y)$ and $E_2 = E \cap [1-y, 1)$. Then $E_1 \cupdot E_2 = E$ and as each is the intersection of measurable sets, $E_1$ and $E_2$ are measurable. Then $m(E) = m(E_1) + m(E_2)$. We have $E_1 \dot{+} y = E_1 + y$ since for each $x \in E_1$, $x< 1-y \implies x+y < 1$ and $x \dot{+} y = x+y$. Then $E_1 \dot{+} y$ is measurable and $m(E\dot{+} y) = m(E+y) = m(E)$ as $m$ is translation invariant. We have $E_2 \dot{+} y = E_2 + (y-1)$ since $1-y < x \implies 1<x+y$ for each $x \in E_2$ so that $x\dot{+}y = x+y - 1$. By Problem 9, $E_2 \dot{+} y = E_2 + (y-1)$ is measurable and $m(E_2 \dot{+} y) = m(E_2 + (y-1)) = m(E_2)$ as $m$ is translation invariant. Then $E\dot{+} y = (E_1 \dot{+} y)\cupdot (E_2 \dot{+} y)$ is measurable as the union of measurable sets and

$$m(E\dot{+} y) = m(E_1\dot{+} y) + m(E_2 \dot{+} y) = m(E_1) + m(E_2) = m(E) \;.$$

{\bf Theorem 17} If $m$ is a countably additive, translation invariant measure defined on a $\sigma$-algebra containing the set $P$, then $m([0,1))$ is either zero or infinite. Since the Lebesgue measure of $[0,1)$ is one, this means $P$ cannot be measurable. \\

Proof: Say that $x \sim y$ if $x - y \in \mathbb{Q}$. This is an equivalence class partitioning $[0,1)$. By the axiom of choice there is a set $P$ which contains one element from each equivalence. Let $(r_i)_{i=0}^\infty$ enumerate the rationals in $[0,1)$ with $r_0 = 0$ and define $P_i = P + r_i$. Then $P_0 = P + 0 = P$. If $x \in P_i\cap P_j$ then $x = p_i \dot{+} r_i = p_j \dot{+} r_j$ and in any case $p_i - p_j \in \mathbb{Q}$ so that $p_i \sim p_j$. If $p_i$ and $p_j$ are in the same equivalence class this means $P_i = P_j$. This implies $(P_i)$ is a pairwise disjoint sequence of sets. Each real number $x \in [0,1)$ must be in some equivalence class (at minimum $x$ is equivalent to $x$ so either $x \in P$ or there is some $y \in P$ such that $y \sim x$). This means $x$ differs from some element of $P$ by a rational number in $[0,1)$. Since $(r_i)$ is an enumeration of all the rationals in $[0,1)$, $x = y + r_i$ for some $y \in P$ so that $x \in P_i \subset \bigcup_{i} P_i$. Since $x \in [0,1)$ was arbitrary, $[0,1) \subset \bigcup P_i$. If $x \in \bigcup P_i$, $x \in P\dot{+} r_i$ so that $x = y \dot{+} r_i$ for some $y \in P$. This means that $x \in [0,1)$ as $y\dot{+} r_i$ is defined by modular addition so that the sum always returns to the interval $[0,1)$. If $P$ is indeed measurable, then by the translation invariance and countable additivity of $m$,
$$m([0,1)) = m\left(\bigcup P_i\right) = \sum m(P_i) = \sum m(P) \;.$$

If $m(P) = 0$, this shows that $m([0,1)) = 0$. If $m(P) > 0$, this shows that $m([0,1)) = \infty$. In particular, this contradicts $m([0,1))$ if $m$ is the Lebesgue measure, so $P$ cannot be measurable. \\

{\bf Corollary to Theorem 17} If $m$ is a countably additive, translation invariant measure with $m([0,1)) = 1$, then $P$ is not measurable. \\

Proof: Suppose $P$ is measurable. Then by Theorem 17, $m([0,1)) \neq 1$. \\



{\bf Problem 17}\\

a. Let $P$ be the nonmeasurable set from Theorem 17. Suppose that $m^*(A\cap P) + m^*(A \cap P^c) \leq m^*(A)$ for each $A \subset \mathbb{R}$. This would give the contradiction that $P$ is measurable. So there must exist an $A \subset \mathbb{R}$ such that $m^*(A \cap P) + m^*(A \cap P^c) > m^*(A)$. Set $E_1 = A\cap P$, $E_2 = A \cap P^c$ and $E_i = \emptyset$ for $i \geq 3$.

$$m^*\left(\bigcup E_i\right) = m^*(A) < m^*(A \cap P) + m^*(A \cap P^c) = \sum m^*(E_i) \;.$$


b.\\

\subsection*{5 Measurable Functions}

{\bf Proposition 18} Let $f$ be an extended real-valued function whose domain is measurable. The following statements are equivalent:\\


i. $\; \; \,$ For each real number $\alpha$ the set $\{x : f(x) > \alpha \}$ is measurable.\\
ii. $\; \,$ For each real number $\alpha$ the set $\{x : f(x) \geq \alpha \}$ is measurable.\\
iii. $\,$ For each real number $\alpha$ the set $\{x : f(x) < \alpha \}$ is measurable.\\
iv. $\; $ For each real number $\alpha$ the set $\{x : f(x) \leq \alpha \}$ is measurable.\\
These statements imply\\
v. $\; \; \, $For each real number $\alpha$ the set $\{x : f(x) = \alpha\}$ is measurable. \\

Proof: Let $D$ be the measurable domain of $f$. If (i) holds, then $\{x : f(x) > \alpha \}^c = \{x : f(x) \leq \alpha\}$, which implies (iv). Similarly, (iv) implies (i), (iii) implies (ii) and (ii) implies (iii). \\

(i) $\implies$ (ii)\\
Let $\alpha$ be given. If (i) holds then as $\alpha - 1/n$ is also a real number, we have $\{x: f(x) > \alpha - 1/n \}$ measurable for each $n$. Since $\mathfrak{M}$ is a $\sigma$-algebra, the intersection $\cap_{n=1}^\infty \{x : f(x) > \alpha - 1/n\}$ is measurable. If $y$ is in this intersection then $f(y) > \alpha - 1/n$ for all $n$, so that $f(y) \geq \alpha$ and $y \in \{x: f(x) \geq \alpha\}$. If $y \in \{x : f(x) \geq \alpha\}$ then $f(y) \geq \alpha > \alpha - 1/n$ for every $n$, so $y \in \cap_{n=1}^\infty \{x: f(x) > \alpha - 1/n\}$. This shows that $\{x : f(x) \geq \alpha \} = \cap_{n=1}^\infty \{x : f(x) > \alpha - 1/n \}$ is measurable. Since $\alpha$ was arbitrary, conclude that (i) implies (ii).\\

(iii) $\implies$ (iv)\\
Let $\alpha$ be given and assume (iii) holds. For each $n$, $\{x : f(x) < \alpha + 1/n \}$ is measurable and similar to the previous argument, this implies that $\{x : f(x) \leq \alpha \} = \cap_{n=1}^\infty \{x : f(x) < \alpha + 1/n\}$ is measurable. Since $\alpha$ was arbitrary conclude that (iii) implies (iv). \\

From these arguments, statements (i)-(iv) are equivalent. We could also make a bidirectional loop by showing (ii) implies (i) and (iv) implies (iii). If (ii) holds and $\alpha$ is arbitrary, then $\cup_{n=1}^\infty \{x : f(x) \geq \alpha + 1/n\} = \{x: f(x) > \alpha\}$ is measurable as the union of measurable sets. From this conclude that (ii) $\implies $ (i). Similarly, use $\{x : f(x) < \alpha\} = \cup_{n=1}^\infty \{x : f(x) \leq \alpha - 1/n\}$ to show that (iv) $\implies$ (iii). \\

Suppose any of (i)-(iv) hold and suppose $\alpha = \infty$. Then (i) must hold. Note that $\{x : f(x) = \infty \} = \cap_{n=1}^\infty \{x : f(x) > n\}$ is measurable as the intersection of measurable sets. We could also have used (ii) $\implies$ (v). \\

Suppose any of (i)-(iv) hold and suppose $\alpha = -\infty$. Then (iii) must hold and $\{x : f(x) = -\infty\} = \cap_{n=1}^\infty \{x : f(x) < n\}$ is measurable as the intersection of measurable sets. We could also have used (iv) $\implies$ (v). \\

Suppose any of (i)-(iv) hold and suppose $\alpha \in \mathbb{R}$. Then (ii) and (iv) must hold and $\{x: f(x) = \alpha\} = \{x : f(x) \geq \alpha \} \cap \{x : f(x) \leq \alpha \}$ is measurable as the intersection of two measurable sets. \\

{\bf Definition} A function $f: D \rightarrow \overline{\mathbb{R}}$ is said to be (Lebesgue) measurable if $D \subset \mathbb{R}$ is measurable and $f$ satisfies one of statements (i)-(iv) in Proposition 18. \\

{\bf Proposition 19} Let $c$ be a constant and $f$ and $g$ two measurable real-valued functions on the same domain $D$ (which must be measurable by the definition above). Then the functions $f+c$, $cf$, $f+g$, $g-f$, and $fg$ are also measurable. \\

Proof:\\

Let $\alpha \in \mathbb{R}$ be given. By the previous definition and proposition 18, the set $\{x : f(x) < \alpha - c\}$ is measurable. But this set is the same set as $\{x: f(x) + c < \alpha\}$. Since $\alpha$ was arbitrary, conclude that $\{x: f(x) + c < \alpha\}$ is measurable for any $\alpha \in \mathbb{R}$. By the definition of a measurable function, $f+c$ is measurable. \\

Let $\alpha \in \mathbb{R}$ be given. If $c=0$, then $cf \equiv 0$. Since we are free to choose between conditions (i)-(iv) of Proposition 18, use (iv). If $0 > \alpha$, then $\{x : cf(x) \leq \alpha\} = \{x : 0\leq \alpha\} = \emptyset \in \mathfrak{M}$. If $0\leq \alpha$, $\{x : cf(x) \leq \alpha\} = \{x : 0 \leq \alpha\} = D \in \mathfrak{M}$. In the case that $c \neq 0$, the set $\{x : cf(x) \leq \alpha\} = \{x : f(x) \leq \alpha/c\}$ if $c$ is positive and so measurable by hypothesis that $f$ is measurable. Otherwise $\{x: cf(x) \leq \alpha\} = \{x : cf(x) \geq \alpha/c\}$  if $c$ is negative, which again gives a measurable set by hypothesis. Therefore $cf$ is measurable. \\

Let $\alpha \in \mathbb{R}$ be given. For any $x$ such that $f(x) + g(x) < \alpha$, there is a rational number $r$ such that $f(x) < r <\alpha - g(x)$. Then $f(x) < r$ and $g(x) < \alpha - r$. So $x \in \{x : f(x) < r\}\cap\{x : g(x) < \alpha - r\} \subset \bigcup_{r\in \mathbb{Q}} \left(\{x : f(x) < r\}\cap\{x : g(x) < \alpha - r\}\right)$ Conversely, if $x \in \bigcup_{r\in \mathbb{Q}} \left(\{x : f(x) < r\}\cap\{x : g(x) < \alpha - r\}\right)$, then  $x \in \{x : f(x) < r\}\cap\{x : g(x) < \alpha - r\}$ for some $r \in \mathbb{Q}$. This means $f(x) < r$ and $g(x) < \alpha - r$ and $f(x) + g(x) < r + g(x) < r + \alpha - r = \alpha$. Therefore $x \in \{x : (f+g)(x) < \alpha\}$. This shows:

$$\{x : (f+g)(x) < \alpha\} = \bigcup_{r \in \mathbb{Q}} \left(\{x : f(x) < r\}\cap\{x : g(x) < \alpha - r\}\right)\;.$$

Since $\{x: f(x) < r\}$ and $\{x: g(x) < \alpha - r\}$ are measurable, so is their intersection by properties of $\sigma$-algebras. A countable union of measurable sets is also measurable by the definition of a $\sigma$-algebra. Conclude that for any $\alpha$, $\{x : (f+g)(x) < \alpha\}$ is measurable so that $f+g$ is a measurable function.\\

Since $g$ is measurable and $(-1)f$ is measurable, $g-f = (-1)f + g$ is also measurable by the previous two paragraphs of this proof. \\

Let $\alpha \in \mathbb{R}$ be given. If $\alpha < 0$, the set $\{x : f^2(x) > \alpha\} = D \in \mathfrak{M}$. If $\alpha \geq 0$, $\{x : f^2(x) > \alpha\} = \{x : f(x) > \sqrt{\alpha}\} \cup \{x : f(x) < -\sqrt{\alpha}\}$ is measurable as the union of two measurable sets. This shows that $f^2$ is measurable. Similarly, $g^2$ is measurable. Then $(f+g)^2$, $-f^2$, and $-g^2$ are measurable and

$$fg  = \frac{1}{2}2fg = \frac{1}{2}[f^2 + 2fg + g^2 -f^2 - g^2] = \frac{1}{2}[(f+g)^2 - f^2 - g^2] \text{ is a measurable function}.$$



{\bf Theorem 20} Let $(f_n)$ be a sequence of measurable functions, each defined on the same measurable domain $D$. Then the functions $\text{sup}\{f_1,...,f_n\}$, $\text{inf}\{f_1,...,f_n\}$, $\text{sup}_{n} f_n$, $\text{inf}_n f_n$, $\text{lim sup } f_n$, and $\text{lim inf } f_n$ are all measurable. \\

Proof: \\

Let $\alpha \in \mathbb{R}$ be given and let $h =\text{sup}\{f_1,...,f_n\}$, i.e. $h(x) = \text{sup}\{f_1(x),...,f_n(x)\}$. If $x \in \{x : h(x) > \alpha\}$. Then by definition of supremum, $f_i(x) > \alpha$ for some $i \in \{1,...,n\}$. If not, $f_i(x) \leq \alpha$ for all $i$ so that $\alpha$ is an upper bound of $\{f_1(x),...,f_n(x)\}$ and $h(x) \leq \alpha$, giving a contradiction. So $x \in \bigcup_{i=1}^n \{x : f_i(x) > \alpha\}$. Conversely, if $x \in \bigcup_{i=1}^n \{x : f_i(x) > \alpha\}$, $f_i(x) > \alpha$ for some $i$ and so $h(x) \geq f_i(x) > \alpha$. Then $x \in \{x : h(x) > \alpha\}$. Since each $\{x : f_i(x) > \alpha\}$ is a measurable, so is this union. Conclude that $\{x : h(x) > \alpha\}= \bigcup_{i=1}^n \{x : f_i(x) > \alpha\}$ is a measurable set. Since $\alpha$ was arbitrary, conclude that $h$ is a measurable function. \\

Replacing $n \in \mathbb{N}$ with $\infty$ in the preceding paragraph shows that $\text{sup}_n f_n$ is a measurable function. No part of the argument above relies on $n$ being finite, only that the unions of sets are countable.\\

Let $\alpha \in \mathbb{R}$ be given and let $g = \text{inf}\{f_1,...,f_n\}$, i.e. $g(x) = \text{inf}\{f_1(x),...,f_n(x)\}$. If $x \in \{x : g(x) < \alpha\}$. Then by definition of infimum, $f_i(x) < \alpha$ for some $i \in \{1,...,n\}$. If not, $f_i(x) \geq \alpha$ for all $i$ so that $\alpha$ is a lower bound of $\{f_1(x),...,f_n(x)\}$ and $g(x) \geq \alpha$, giving a contradiction. So $x \in \bigcup_{i=1}^n \{x : f_i(x) < \alpha\}$. Conversely, if $x \in \bigcup_{i=1}^n \{x : f_i(x) < \alpha\}$, then $f_i(x) < \alpha$ for some $i$ and $g(x) \leq f_i(x) < \alpha$ so that $x \in \{x : g(x) < \alpha\}$. Conclude that $\{x : g(x) < \alpha\}= \bigcup_{i=1}^n \{x : f_i(x) < \alpha\}$ is a measurable set. Since $\alpha$ was arbitrary, conclude that $g$ is a measurable function.\\

Similar to the case with supremum, replacing $n \in \mathbb{N}$ with $\infty$ in the preceding paragraph shows that $\text{inf}_n f_n$ is a measurable function. No part of the argument above relies on $n$ being finite, only that the unions of sets are countable.\\

By what has been established so far, it follows that for each $n \in \mathbb{N}$, $\text{sup}_{n\geq k} f_k$ is a measurable function. This implies that $\text{inf}_{n} \left(\text{sup}_{k\geq n} f_k \right) =: \text{ lim sup } f_n$ is a measurable function. Similarly that $\text{sup}_{n} \left(\text{inf}_{k\geq n} f_k \right) =: \text{ lim inf } f_n$ is measurable. \\

{\bf Definition} A property is said to hold \underline{almost everywhere} if the set of points where it fails to hold is a set of measure 0. \\

{\bf Proposition 21} If $f$ is a measurable function and $f = g$ almost everywhere ($f$ and $g$ have the same domain and $m\{x : f(x) \neq g(x)\} = 0$, then $g$ is measurable. \\

Proof: Let $\alpha \in \mathbb{R}$ be given and let $E = \{x : f(x) \neq g(x)\}$. By hypothesis, $m(E) = 0$. The set $\{x : f(x) > \alpha\}$ is measurable as $f$ is a measurable function. The set $\{x: x \in E \text{ and } g(x) > \alpha \}$ is a subset of $E$ and contains the points such that $g(x) > \alpha$ but $g(x) \neq f(x)$. As $m^*(E) = m(E) = 0$, $m^*(\{x: x \in E \text{ and } g(x) > \alpha \}) = 0$. By Lemma 6, $\{x: x \in E \text{ and } g(x) > \alpha \}$ is measurable (any set of real numbers with outer measure 0 is measurable). The set $\{x: x \in E \text{ and } g(x) \geq \alpha\}$ is a subset of $E$ and contains the points $x \in E$ such that $g(x) \leq \alpha$ and $f(x) \neq g(x)$. This set must also have outer measure zero as a subset of $E$, so conclude that $\{x: x \in E \text{ and } g(x) \leq \alpha \}$ and hence $\{x: x \in E \text{ and } g(x) \leq \alpha \}^c$ are measurable sets. If we can show set equality,

$$\{x : g(x) > \alpha\} = \left[\{x : f(x) > \alpha\} \bigcup \{x: x \in E \text{ and } g(x) > \alpha \}\right] \bigcap \{x: x \in E \text{ and } g(x) \leq \alpha \}^c$$

is measurable and therefore $g$ is a measurable function as $\alpha$ was arbitrary. We prove next that these sets are the same.\\

Let $x \in \{x : g(x) > \alpha\}$. Either $g(x) \neq f(x)$ or $g(x) = f(x)$. If $g(x) \neq f(x)$, then $x \in E$ and $g(x) > \alpha$. Then $x \in \{x : x \in E \text{ and } g(x) > \alpha\} \subset \{x : f(x) > \alpha\} \bigcup \{x: x \in E \text{ and } g(x) > \alpha \}$. Since $g(x) \not\leq \alpha$, $x \in \{x: x \in E \text{ and } g(x) \leq \alpha\}^c$, which completes the forward inclusion for $x \in E$. If then $f(x) = g(x)$, then $x \not\in E$. Since $g(x) > \alpha$, $f(x) > \alpha$. Since $x\not\in E$, $x \not\in \{x : x \in E \text{ and } g(x) \leq \alpha\}$. This completes the forward inclusion for $x \not\in E$.\\

If $x \in \left[\{x : f(x) > \alpha\} \bigcup \{x: x \in E \text{ and } g(x) > \alpha \}\right] \cap \{x: x \in E \text{ and } g(x) \leq \alpha \}^c$, consider again whether $f(x) = g(x)$ or $f(x) \neq g(x)$. If $f(x) \neq g(x)$, then $x \in E$. We have that $x \in \{x : x \in E \text{ and } g(x) \leq \alpha\}^c$. Either $x \not\in E$ of $x \in E$ but $g(x) > \alpha$. Since we assumed $x \in E$, it must be the case that $g(x) > \alpha$. Thus $x \in \{x : g(x) > \alpha\}$. If $f(x) = g(x)$, then $x \not\in E$. But since $x \in \{x : f(x) > \alpha\} \bigcup \{x : x \in E \text{ and } g(x) > \alpha\}$, it must be that $f(x) > \alpha$. But this means $g(x) > \alpha$, so $x \in \{x  : g(x) > \alpha\}$. Since we again have condition on the two possible cases that $f(x) = g(x)$ or $f(x) \neq g(x)$, conclude that the reverse inclusion holds as well. \\

{\bf Proposition 22} Let $f$ be a measurable function defined on an interval $[a,b]$, and assume that $f$ takes on the values $\pm \infty$ only on a set of measure zero. Then given $\epsilon$, we can find a step function $g$ and a continuous function $h$ such that

$$|f-g| < \epsilon \text{ and } |f-h| < \epsilon $$

except on a set of measure less than $\epsilon$; i.e., $m(\{x : |f(x) - g(x)| \geq \epsilon \}) < \epsilon$ and $m(\{x : |f(x) - g(x)| \geq \epsilon \}) < \epsilon$. If in addition, $m \leq f \leq M$, then we may choose the functions $g$ and $h$ such that $m \leq g,h \leq M$. \\

{\bf Definition} If $A$ is any set, we define the \underline{characteristic function $\chi_A$ of the set $A$} as

$$ \chi_A(x) = \begin{cases}
1 & x \in A\\
0 & x \not\in A
\end{cases} \;.$$

The function $A$ is measurable if and only if $A$ is measurable.\\

{\bf Remark} By this definition, the existence of a nonmeasurable set implies the existance of a nonmeasurable function. \\

{\bf Definition} A real-valued function $\phi$ is called \underline{simple} if it is measurable and assumes only a finite number of values. If $\phi$ is simple and has the values $\alpha_1,...,\alpha_n$ then $\phi = \sum_{i=1}^n \alpha_i \chi_{A_i}$ where $A_i = \{x : \phi(x) = \alpha_i\}$. The sum, product, and difference of two simple functions are simple. \\

{\bf Problem 18} Show that (v) does not imply (iv) in Proposition 18 by constructing a function $f$ such that $\{x : f(x) > 0 \} = E$, a given nonmeasurable set, and such that $f$ assumes each value at most once. \\

Let $E$ be a given nonmeasurable set. The existence of a nonmeasurable set comes from section 4. Define the function $f: \mathbb{R} \rightarrow \overline{\mathbb{R}}$ by

$$f(x) = \begin{cases} 2^x & x \in E \\ -2^x & x \not\in E \end{cases} \;. $$

Then $f$ is an extended real-valued function whose domain is measurable. We will show that part (v) of Proposition 18 holds yet part (iv) fails. \\

If $f(x) = f(y)$, then since $2^z > 0$ for all $z$ and $-2^z < 0$ for all $z$, this means that either $2^x = 2^y$ or $-2^x = -2^y$. In either case, $x = y$. By this construction, $f(x) > 0$ if $x \in E$ and $f(x) < 0$ if $x \not\in E$ so that $\{x : f(x) > 0 \} = E$ is nonmeasurable. By the injectivity of $f$, for each $\alpha \in \overline{\mathbb{R}}$, $\{x : f(x) = \alpha\}$ is either a singleton or empty. In either situation, $m^*(\{x : f(x) = \alpha \}) = 0$. This means for each $\alpha \in \overline{\mathbb{R}}$, $\{x : f(x) = \alpha \}$ is measurable (Lemma 6 or Theorem 10). However, for $\alpha = 0$, if the set $\{x : f(x) \leq 0\}$ were a measurable set, then $\{x : f(x) \leq 0\}^c = \{x : f(x) > 0\} =E$ must also be measurable as $\mathfrak{M}$ is a $\sigma$-algebra (Theorem 10). This is a contradiction so conclude that although (v) holds, there exists an $\alpha \in \mathbb{R}$ such that $\{x : f(x) \leq \alpha\}$ is not measurable so that (iv) does not necessarily follow from (v). \\

{\bf Problem 19} Let $D$ be dense in $\mathbb{R}$ and let $f: \mathbb{R} \rightarrow \overline{\mathbb{R}}$  such that $\{x : f(x) > \alpha\}$ is measurable for each $\alpha \in D$. Prove that $f$ is measurable. \\

If $D = \mathbb{R}$ then $f$ is measurable by immediate application of Proposition 18 (i) and the definition of a Lebesgue measurable function. If $D \neq \mathbb{R}$,  consider $\alpha \in \mathbb{R} \backslash D$. As $D$ is dense in $\mathbb{R}$, for each $n \in \mathbb{N}$, $D \cap (\alpha, \alpha + 1/n) \neq \emptyset$. For each $n$ pick an element $d_n \in D\cap (\alpha, \alpha + 1/n)$ to construct the sequence $(d_n)$. Since $\mathfrak{M}$ is a $\sigma$-algebra, each set $\{x : f(x) \leq d_n\} = \{x : f(x) > d_n\}^c$ is measurable and  the countable intersection of measurable sets $\cap_{n=1}^\infty \{x : f(x) \leq d_n \}$ is measurable. Let us prove that

$$\{x : f(x) \leq \alpha\} = \bigcap_{n=1}^\infty \{x : f(x) \leq d_n \} \;.$$

If $y \in \{x : f(x) \leq \alpha\}$, then $f(y) \leq \alpha< d_n$ for each $n \in \mathbb{N}$. Since $f(y) < d_n$, $f(y) \leq d_n$ so $y \in \{x : f(x) \leq  d_n\}\subset \cap_{n=1}^\infty \{x : f(x) \leq d_n\}$. If $y \in \cap_{n=1}^\infty \{x : f(x) \leq d_n\}$, then $f(y) \leq d_n$ for each $n$. Suppose that $f(y) > \alpha$. Since $\alpha < d_n < \alpha + 1/n$ for each $n$, $(d_n)$ converges to $\alpha$ (note that $D$ cannot be closed if $D \neq \mathbb{R}$ so we are not in danger of the contradiction $\alpha \in D$ here). This means there is an $N$ such that  $f(y) > d_n > \alpha$ for each $n \geq N$, which contradicts $f(y) \leq d_n$ for each $n$. So $f(y) \leq \alpha$ and $y \in \{x: f(x) \leq \alpha\}$. Conclude that $\{x : f(x) \leq \alpha\}$ is measurable for each $\alpha \in \mathbb{R}\backslash D = D^c$.  Since $\{x : f(x) > \alpha\}$ is measurable for each $\alpha \in D$, each complement $\{x : f(x) \leq \alpha\}$ is measurable for each $\alpha \in D$. Therefore, $\{x : f(x) \leq \alpha\}$ is measurable for each $\alpha \in \mathbb{R}$. By Proposition 18 (iv) and the definition of a Lebesgue measurable function. \\

{\bf Problem 20} \\

(Part 1) Show that the sum of simple functions is a simple function and the product of simple functions is a simple function. \\

Let $\phi_1$ and $\phi_2$ be simple functions both defined on the measurable set $E \subset \mathbb{R}$ (by definition, a simple function must be measurable which means its domain must be measurable) with:

\begin{align*}
\phi_1(x) &= \sum_{n = 1}^N \alpha_n \chi_{A_n}, \quad \text{ where } A_n = \{x : \phi_1(x) = \alpha_n \} \\
\phi_2(x) &= \sum_{m = 1}^M \beta_m \chi_{B_m}, \quad \text{ where } B_m = \{x : \phi_2(x) = \beta_m \} \\
\end{align*}

Since $\phi_1$ and $\phi_2$ are measurable functions, the sum $\phi_1 +\phi_2$ is measurable. To show that $\phi_1 + \phi_2$ is simple, we need to show that $\phi_1 + \phi_2$ assumes only a finite number of values. Adding sets and relabeling if necessary, we can assume that $E = \cupdot A_m$ and $E = \cupdot$. This allows us to write:

\begin{align*}
\phi_1 + \phi_2 &= \sum_{m = 1}^M \sum_{n=1}^N (\alpha_n + \beta_m) \chi_{A_n \cap B_m} \\
&= \sum_{m=1}^M \left[(\alpha_1 + \beta_m)\chi_{A_1\cap B_m} + ... + (\alpha_N + \beta_m)\chi_{A_N\cap B_m}\right] \\
&= \left[(\alpha_1 + \beta_1)\chi_{A_1\cap B_1} + ... + (\alpha_N + \beta_1)\chi_{A_N\cap B_1}\right] \\
&+ \left[(\alpha_1 + \beta_2)\chi_{A_1\cap B_2} + ... + (\alpha_N + \beta_2)\chi_{A_N\cap B_2}\right] \\ 
&\vdots \\
& + \left[(\alpha_1 + \beta_M)\chi_{A_1\cap B_M} + ... + (\alpha_N + \beta_M)\chi_{A_N\cap B_M}\right] \\
\end{align*}
\begin{align*}
&:= [\gamma_1 \chi_{C_1} + ... + \gamma_N \chi_{C_N} ]\\
&+ [\gamma_{N+1} \chi_{C_{N+1}} + ... + \gamma_{2N} \chi_{C_{2N}} ]\\
& \vdots \\
& + [\gamma_{(M-1)N + 1} \chi_{C_{(M-1)N + 1}} + ... + \gamma_{MN} \chi_{C_{MN}}] \\
&= \sum_{i = 1}^{MN} \gamma_i \chi_{C_i} \; ,
\end{align*}

where we used the fact that for any pair $n,m$, $\alpha_n + \beta_m$ is a real number and $A_{n} \cap B_{m}$ is a measurable set so that we can we can assign a coefficient $\gamma_i$ and a characteristic function $\chi_{C_i}$ to each term in the finite sum. Also, since the $A_n$ and $B_m$ are disjoint, $(A_n \cap B_m) \cap (A_{n'} \cap B_{m'}) = \emptyset$ whenever $(n,m) \neq (n',m')$. This shows that $\phi_1 + \phi_2$ can be written in the form of a simple function and so takes on only a finite number of values. Conclude that $\phi_1+\phi_2$ is simple. By very similar reasoning, using coefficients of the form $\gamma_i = \alpha_n \beta_m$ instead of $\gamma_i = \alpha_n + \beta_m$ for the possible pairs $(n,m)$,

$$\phi_1 \phi_2 = \sum_{m=1}^M \sum_{n  = 1}^N \alpha_n\beta_m \chi_{A_n \cap B_m} = \sum_{i = 1}^{MN} \gamma_i \chi_{C_i} \;,$$

where the $C_i$ are disjoint measurable sets with $\cupdot C_i = E$. Conclude that $phi_1 \phi_2$ is a measurable function (as the product of measurable functions) that assumes only a finite number of values and is therefore a simple function. \\

(Part 2) Let $A$ and $B$ be sets of real numbers and $\chi_A$ and $\chi_B$ corresponding characteristic functions. Show that the sum $\chi_A + \chi_B$ and the product $\chi_A \chi_B$ are simple functions and that,

\begin{align*}
\chi_{A\cap B} &= \chi_A \chi_B  \\
\chi_{A\cup B} &= \chi_A + \chi_B - \chi_A \chi_B \\
\chi_{A^c} &= 1 - \chi_{A}
\end{align*}

We may as well assume that $\chi_A, \chi_B, \chi_{A\cap B}, \chi_{A\cup B}, \chi_{\overline{A}} : D\subset \mathbb{R} \rightarrow \{0,1\}$; that is, all characteristic functions mentioned have the same domain $D$ where $A,B \subset D$. The reasoning would look the same for any choice of $D \subset \mathbb{R}$. Whether or not $D$ is measurable is not important for the identities we are proving here, only for whether the functions are measurable or not. Let $x \in D$.\\

Either $x \in A\cap B$ or $x \not\in A\cap B$. If $x\in A\cap B$ then $\chi_{A\cap B}(x) = 1$, $\chi_A(x) = 1$, and $\chi_B(x) = 1$ so $\chi_{A\cap B}(x) = 1 = 1\cdot 1 = \chi_{A}(x) \chi_{B}(x)$. If $x \not\in A\cap B$ then $\chi_{A\cap B}(x) = 0$ and $\chi_{A}(x) = 0$ or $\chi_{B}(x) = 0$ (or both of course). Then $\chi_{A\cap B}(x) = 0 = \chi_{A}\chi_{B}$. Since $x$ was arbitrary, this shows that $\chi_{A\cap B} \equiv \chi_{A}\chi_{B}$ on $D$. \\

Exactly one of these four possibilities must hold: (i) $x \in A\cap B$, (ii) $x \in A\backslash B$, (iii) $x \in B\backslash A$, or (iv) $x \not\in A\cup B$. We check that the equality $\chi_{A\cup B}(x) = \chi_A(x) + \chi_B(x) - \chi_{A\cap B}(x)$ is satisfied in each case. \begin{align*}
&\text{(i) } \chi_{A\cup B}(x) = \chi_{A}(x) = \chi_{B}(x)  = \chi_{A\cap B}(x) = 1\\
&\implies \chi_{A\cup B}(x) = 1 = 1+1 - 1 = \chi_{A}(x) + \chi_{B}(x) - \chi_{A\cap B}(x) \\
&\text{(ii) } \chi_{A\cup B}(x) = \chi_{A}(x) = 1, \chi_{B}(x)  = \chi_{A\cap B}(x) = 0\\
&\implies \chi_{A\cup B} = 1 = 1+0 - 0 = \chi_{A}(x) + \chi_{B}(x) - \chi_{A\cap B}(x) \\
&\text{(iii) } \chi_{A\cup B}(x) = \chi_{B}(x) = 1, \chi_{A}(x)  = \chi_{A\cap B}(x) = 0 \\
&\implies \chi_{A\cup B} = 1 = 0+1 - 0 = \chi_{A}(x) + \chi_{B}(x) - \chi_{A\cap B}(x) \\
&\text{(iv) } \chi_{A\cup B}(x) = \chi_{A}(x) = \chi_{B}(x)  = \chi_{A\cap B}(x) = 0 \\
&\implies \chi_{A\cup B}(x) = 0 = 0+0-0= \chi_{A}(x) + \chi_{B}(x) - \chi_{A\cap B}(x) \\
\end{align*}

Either $x \in A$ or $x \in A^c$. If $x \in A$, then $\chi_{A^c}(x) = 0$ and $\chi_{A} = 1$ so $\chi_{A^c} = 0 = 1-1 = 1-\chi_{A}$. If $x \in A^c$, then $\chi_{A^c} = 1$ and $\chi_{A} = 0$ so $\chi_{A^c} = 1 = 1 - 0 = 1-\chi_{A}$. Conclude that $\chi_{A^c} \equiv 1- \chi_{A}$ on $D$. 


{\bf Problem 21}\\

a. Let $E$ and $D$ be measurable sets and $f$ a function with domain $E\cup D$. Show that $f$ is measurable if and only if its restrictions to $D$ and $E$ are measurable. \\

Suppose that $f$ is measurable and let $\alpha \in \mathbb{R}$ be given. Then the set $\{x \in E\cup D : f(x) < \alpha\}$ is measurable (the notation needs to be more explicit here so we can consider whether $x \in D$ or $x \in E$). Consider that

$$\{x \in D : f\rvert_D(x) < \alpha\} = \{x \in D : f(x) < \alpha\} = \{x \in E\cup D : f(x) < \alpha\} \cap D \;.$$

Then $\{x \in D : f(x) < \alpha\}$ is measurable as the intersection of two measurable sets. Since $\alpha$ was arbitrary, conclude that $\{x \in D : f\rvert_D(x) < \alpha\}$ is measurable for any $\alpha$. Conclude that the restriction of $f$ to $D$, $f\rvert_D$, is a measurable function. By swapping the positions of $D$ and $E$, the same reasoning shows that the restriction of $f$ to $E$, $f\rvert_E$, is a measurable function as well. \\

Suppose that $f\rvert_D$ and $f\rvert_E$ are measurable functions and let $\alpha \in \mathbb{R}$ be given. Since $E$ and $D$ are measurable, the domain $E\cup D$ of $f$ is measurable and

\begin{align*}
\{x \in E\cup D : f(x) < \alpha\} &= \{x \in E : f(x) < \alpha\}\cup \{x \in D : f(x) < \alpha\}\\
&= \{x \in E : f\rvert_E(x) < \alpha\}\cup \{x \in D : f\rvert_D(x) < \alpha\}\;.
\end{align*}

Since $\{x \in E : f\rvert_E(x) < \alpha\}$ and $\{x \in D : f\rvert_D(x) < \alpha\}$ are measurable sets, so is their union. Since $\alpha$ was arbitrary, $\{x \in E\cup D : f(x) < \alpha\}$ is measurable for each $\alpha$. Therefore, $f$ is a measurable function. \\

b. Let $f$ be a function with a measurable domain $D$ and let

$$g(x) = \begin{cases} f(x) & x \in D\\ 0 & x \not\in D \end{cases} \;.$$

Show that $f$ is measurable if and only if $g$ is measurable.\\

The domain of $g$ has not been specified in the problem.  If the domain of $g$ is not measurable, then by definition it is impossible for $g$ to be measurable and this problem cannot be completed. So we need to assume that the domain $E$ of $g$ is measurable. If $E = D$ then the result is immediate. If $E \subsetneq D$, then it would seem the definition of $g$ does not make sense and also it would not be possible to show that the measurability of $g$ implies the measurability of $f$. So most likely we are to assume that $E \supsetneq D$. Since the reasoning is similar for any such measurable $E$, just assume that the domain of $g$ is $\mathbb{R}$ (which is measurable in what follows. While the work below is still correct if $D = \mathbb{R}$, the result would again be immediate in this case and so this is meant to handle a measurable $D \subsetneq \mathbb{R}$. \\

Suppose that $f$ is measurable and let $\alpha \in \mathbb{R}$ be given. 

\begin{align*}
\{x : g(x) < \alpha\} &= \{x \in D : g(x) < \alpha\} \cup \{x \not\in D : g(x) < \alpha\} \\
&= \{x \in D : f(x) < \alpha\} \cup \{x \in D^c : 0 < \alpha\} \;. 
\end{align*}

Since $f$ is measurable, the set $\{x \in D : f(x) < \alpha\}$ is always measurable, so it remains for us to see if $D^c \cap \{x : g(x) < \alpha\} = \{x  \in D^c : g(x) < \alpha\} = \{x \in D^c : 0< \alpha\}$ is measurable. Either $0 < \alpha$ or $0\geq \alpha$. If $0 < \alpha$, $\{x \in D^c : 0<\alpha\} = D^c$. If $0 \geq \alpha$, $\{x \in D^c : 0 < \alpha\} = \emptyset$. Since $D^c$ and $\emptyset$ are measurable sets, $\{x \in D^c : 0 < \alpha\}$ in either case. Therefore, $\{x : g(x) < \alpha\}$ is measurable as the union of two measurable sets. \\

Suppose that $g$ is measurable (again with the assumptions mentioned in the first paragraph about the domain of $g$ and $f$). Let $\alpha$ be given. The set $\{x \in \mathbb{R} : g(x) < \alpha\}$ is measurable and $D$ is measurable.

$$\{x \in D : f(x) < \alpha\} = \{x \in D : g(x) < \alpha\} = \{x \in \mathbb{R} : g(x) < \alpha\} \cap D \;.$$

Then $\{x \in D: f(x) < \alpha\}$ is measurable as the intersection of two measurable sets. Since $\alpha$ was arbitrary, $\{x \in D : f(x) < \alpha\}$ is measurable for any $\alpha$. Conclude that $f$ is a measurable function. \\

{\bf Problem 22}\\

a. Let $f : D \rightarrow \overline{\mathbb{R}}$ where $D$ is a measurable set. Let $D_1 = \{x : f(x) = \infty\}$ and $D_2 = \{x : f(x) = -\infty\}$. Show that $f$ is measurable if and only if $D_1$ and $D_2$ are measurable and the restriction of $f$ to $D \backslash (D_1 \cup D_2)$ is measurable. \\

Suppose that $f$ is measurable and let $f^\dagger$ denote the restriction of $f$ to $D \backslash (D_1 \cup D_2)$. Since $\{x : f(x) > n\}$ and $\{x : f(x) < -n\}$ are measurable sets for each $n \in \mathbb{N}$,

$$D_1 = \{x : f(x) = \infty\} = \cap_n \{x : f(x) > n\} \in \mathfrak{M},$$
$$D_2 = \{x : f(x) = -\infty\} = \cap_n \{x : f(x) < -n \} \in \mathfrak{M} \;.$$

This implies that $D\cap D_1^c\cap D_2^c = D \backslash (D_1\cup D_2)\in \mathfrak{M}$ as well. Let $\alpha \in \mathbb{R}$. Since $f^\dagger$ is only defined for $x \in D\backslash (D_1 \cup D_2)$,

$$\{x: f^\dagger(x) < \alpha\} = \{x: f(x) < \alpha\} \cap (D_1\cup D_2)^c \;.$$

This shows that $\{x : f^\dagger(x) < \alpha\}$ is measurable as the intersection of measurable sets. Since $\alpha$ was arbitrary, $\{x : f^\dagger(x) < \alpha\}$ is measurable for each $\alpha$ and since the domain of $f^\dagger$ is measurable, conclude that $f^\dagger$ is a measurable function. 

Suppose that $f\dagger$ is measurable and that $D_1$ and $D_2$ are measurable sets. Let $\alpha \in \mathbb{R}$. The set $\{x : f^\dagger(x) < \alpha\}$ is measurable and so

\begin{align*}
\{x \in D : f(x) < \alpha \} &= \{x \in D\backslash (D_1 \cup D_2) : f(x) < \alpha \} \cup \{x \in D : f(x) = -\infty \}\\
&= \{x : f^\dagger(x) < \alpha\}\cup D_2 \in \mathfrak{M} \;.
\end{align*}

Since $\alpha$ was arbitrary and $D$ is measurable, conclude that $f$ is measurable.\\

b. Prove that the product of two measurable extended real-values functions is measurable. \\

Let $f,g : D \rightarrow \overline{\mathbb{R}}$, where $D$ is a measurable set on which both $f$ and $g$ and thus $fg$ can be defined. Assume that both $f$ and $g$ are measurable. By part (a), the sets $\{x : f(x) = \infty\}$, $\{x : f(x) = -\infty\}$, $\{x : g(x) = \infty\}$, and $\{x : g(x) = -\infty\}$ are measurable. Also the sets $\{x : f(x) < 0\}$, $\{x : f(x) > 0\}$, $\{x : g(x)  < 0\}$, and $\{x : g(x)  > 0\}$ are measurable since $f$ and $g$ are measurable (and Proposition 18). By repeatedly using the fact that the $\sigma$-algebra $\mathfrak{M}$ is closed under complement and intersection and the conventions from section 2.3 for multiplication in $\overline{\mathbb{R}}$, the set


\begin{align*}
D_1 &:= \{x : (fg)(x) = \infty\}\\
&=[\{x : f(x) = \infty\} \cap \{x : g(x) > 0\}]\\
&\cup [\{x : f(x) = -\infty\} \cap \{x : g(x) < 0\}]\\
&\cup [\{x : g(x) = \infty\} \cap \{x : f(x) > 0\}]\\
&\cup [\{x : g(x) = -\infty\} \cap \{x : f(x) < 0\}]\\
&\cup [\{x : f(x) = \infty\} \cap \{x : g(x) = \infty\}]\\
&\cup [\{x : f(x) = -\infty\} \cap \{x : g(x) = -\infty\}]\\
& \in \mathfrak{M} \;.
\end{align*}

Similarly,

\begin{align*}
D_2 &:= \{x : (fg)(x) = -\infty\}\\
&=[\{x : f(x) = \infty\} \cap \{x : g(x) < 0\}]\\
&\cup [\{x : f(x) = -\infty\} \cap \{x : g(x) > 0\}]\\
&\cup [\{x : g(x) = \infty\} \cap \{x : f(x) < 0\}]\\
&\cup [\{x : g(x) = -\infty\} \cap \{x : f(x) > 0\}]\\
&\cup [\{x : f(x) = \infty\} \cap \{x : g(x) = -\infty\}]\\
&\cup [\{x : f(x) = \infty\} \cap \{x : g(x) = -\infty\}]\\
& \in \mathfrak{M} \;.
\end{align*}

This implies $D_1 \cup D_2$ and $D \backslash (D_1 \cup D_2)$ are measurable. By Problem 21 (a), the restriction of $f$ to $D \backslash (D_1 \cup D_2)$ is measurable. Similarly, the restriction of $g$ to $D \backslash (D_1 \cup D_2)$ is measurable. Moreover, these restrictions are measurable real-valued functions and so by Proposition 19, the restriction of $fg$ to $D \backslash (D_1 \cup D_2)$ is measurable. By part (a) of this problem, since $D_1$ and $D_2$ are measurable and the restriction of $fg$ to $D \backslash (D_1 \cup D_2)$ is measurable, we conclude that $fg$ is measurable. \\

c. If $f$ and $g$ are measurable extended real-valued functions, and $\alpha \in \mathbb{R}$ is fixed, prove that

$$
(f+g)(x) := \begin{cases}
\alpha & f(x) = \infty, \quad g(x) = -\infty \\
\alpha & f(x) = -\infty, \quad g(x) = \infty \\
f(x) + g(x) & \quad \quad \quad \text{otherwise}
\end{cases} $$

is measurable. \\

Let $f,g : D \rightarrow \overline{\mathbb{R}}$. By part (a) the sets $\{x : f(x) = \infty\}$, $\{x : f(x) = -\infty\}$, $\{x : g(x) = \infty\}$, and $\{x : g(x) = -\infty\}$ are measurable. This implies $\{x : -\infty < g(x) < \infty\} = D \backslash (\{x : g(x) = \infty\} \cup \{x : g(x) = -\infty\})$ and similarly $\{x : -\infty < f(x) < \infty\}$ are measurable.

\begin{align*}
D_1 &:= \{x : (f+g)(x) = \infty\}\\
&=[\{x : f(x) = \infty\} \cap \{x : -\infty < g(x) < \infty \}]\\
&\cup [\{x : -\infty < f(x) < \infty \} \cap \{x:  g(x) = \infty\}]\\
&\cup [\{x : f(x) = \infty \}\cap \{x : g(x) = \infty\}]\\
\end{align*}
\begin{align*}
D_2 &:= \{x : (f+g)(x) = -\infty\}\\
&=[\{x : f(x) = -\infty\} \cap \{x : -\infty < g(x) < \infty \}]\\
&\cup [\{x : -\infty < f(x) < \infty \} \cap \{x:  g(x) = -\infty\}]\\
&\cup [\{x : f(x) = -\infty \}\cap \{x : g(x) = -\infty\}]
\end{align*}

Since $\mathfrak{M}$ is a $\sigma$-algebra, $D_1$ and $D_2$ are measurable. By part (a), if we can show that the restriction of $f$ to $D\backslash (D_1 \cup D_2)$,  $h(x) := (f+g)\rvert_{D \backslash (D_1 \cup D_2)}(x)$ is measurable then we can conclude the extended real-valued function $f+g : D \rightarrow \overline{\mathbb{R}}$ is measurable. Let $E:=D \backslash (D_1 \cup D_2)$. Let $\beta \in \mathbb{R}$ be arbitrary. With Proposition 18 (iii) in mind, we want to show that the set $\{x \in E : h(x) := (f+g)(x) < \beta\}$ is measurable. Since $\alpha$ is fixed, consider whether $\alpha < \beta$ or $\alpha \geq \beta$. Let $F = E\cap\{x : -\infty < f(x) < \infty\}\cap\{x : -\infty < g(x) < \infty\} \in \mathfrak{M}$. For $x \in F$, $f(x)$ and $g(x)$ are both finite so that $h(x) = (f+g)(x)$ is a measurable function by Proposition 19. So the set $\{x \in F : h(x) < \beta\}$ is measurable.

\begin{align*}
\{x \in E : h(x) < \beta\} &= \{x \in F : h(x) < \beta\}\\ &\cup [\{x \in E : f(x) = \infty\} \cap \{x \in E : g(x) = -\infty\}]\\
&\cup [\{x : f(x) = -\infty\} \cap \{x : g(x) = \infty\}] \in \mathfrak{M}, \quad \alpha < \beta\\
\{x \in E : h(x) < \beta\} &= \{x \in F : h(x) < \beta\} \in \mathfrak{M}, \quad \alpha \geq \beta \;.
\end{align*}

Note that the sets above of a form like $\{x \in E : f(x) = \infty\} = \{x \in D: f(x) = \infty\}\cap D_1^c \cap D_2^c$ are indeed measurable and that it is necessary to mention these cases as $D_1 \cup D_2$ does not include all possible instances where $f$ and $g$ are infinite. Since $\beta$ was arbitrary, conclude that the restriction of $f+g$ to $D\backslash (D_1 \cup D_2)$ is measurable and so by part (a), the extended real valued function $f+g$ is measurable. \\

(d) Let $f,g : D\rightarrow \overline{\mathbb{R}}$ be measurable extended real-valued functions such that $f$ and $g$ are each finite almost everywhere. Show that $f+g$ is measurable no matter how it is defined at points where it is of the form $\infty - \infty$ (and presumably $-\infty + \infty$).

\begin{align*}
&C_1 := \{x : f(x) = \infty\}\cup \{x : f(x) = -\infty\}\\
&C_1 := \{x : g(x) = \infty\}\cup \{x : g(x) = -\infty\}\\
&m(C_1) = m(C_2) = 0 \text{ by hypothesis.}\\
& 0\leq m(C_1\cup C_2) \leq m(C_1) + m(C_2) = 0 \implies m(C_1\cup C_2) = 0\\
& B:= [\{x : f(x) = \infty\} \cap \{x : g(x) = -\infty\}]\cup [\{x : f(x) = -\infty\} \cap \{x : g(x) = \infty\}]
\end{align*}

The set $B$ is the set of points at which $f+g$ is of the form $\infty - \infty$ or $-\infty + \infty$. Let $f+g$ be defined arbitrarily at points in $B$. To see that $B\subset C_1 \cup C_2$, let $y \in B$. If $y \in  \{x : f(x) = \infty\} \cap \{x : g(x) = -\infty\}$, then $f(y) = \infty$ and $g(y) = -\infty$. So $y \in C_1$ and $y \in C_2$ and $y \in C_1\cap C_2 \subset C_1 \cup C_2$. If $y \in \{x : f(x) = -\infty\} \cap \{x : g(x) = \infty\}$ it follows similarly that $y \in C_1 \cup C_2$. This implies that $m(B) = 0$. Define $h : D \rightarrow \overline{\mathbb{R}}$,

$$h(x) = \begin{cases}
(f+g)(x) & x \in B^c\\
27 & x \in B \end{cases} \;. $$

Then $h$ is measurable and the set of points at which $h \neq f+g$ has measure zero. That is, $h$ is a measurable function and $h = f+g$ almost everywhere. By Proposition 21, conclude that $f+g$ is measurable. \\

{\bf Problem 23} Prove Proposition 22 by establishing the following lemmas:\\

(a) Given a measurable function $f$ on $[a,b]$ that takes on the values $\pm \infty$ only on a set of measure zero, and given $\epsilon > 0$, there is an $M$ such that $|f| \leq M$ except on a set of measure less than $\epsilon / 3$.\\

Let $E_n = \{x : |f(x)| > n\}$ for each $n\in \mathbb{N}$. Since $f$ is measurable, $|f|$ is measurable and therefore each set $E_n$ is measurable. Also, $E_{n+1} \subset E_n$ for each $n$ and $m(E_1) \leq m([a,b]) < \infty$. By Proposition 14,

$$ \text{lim}_{n\rightarrow \infty} m(E_n) =  m\left(\bigcap_{n=1}^\infty  E_n\right) = m(\{x : f(x) = \pm \infty \} ) = 0  \; . $$

It follows that given $\epsilon > 0$, there is an $M$ such that for all $n \geq M$, $m(E_n) < \epsilon /3$. In particular, $m(E_M) < \epsilon / 3$ and $E_M^c = \{x : |f(x)| \leq M\}$. That is, $|f| \leq M$ except on the set $E_M$, which is of measure $\epsilon / 3$. \\

(b) Let $f$ be a measurable function on $[a,b]$. Given $\epsilon > 0$ and $M \geq 0$ there is a simple function $\phi$ such that $|f(x) - \phi(x)| < \epsilon $ except where $|f(x)| \geq M$. If $m \leq f \leq M$, then we may take $\phi$ so that $m \leq \phi \leq M$. \\


Let $\epsilon > 0$. Since $M < \infty$, there is a $k \in \mathbb{N}$ such that $k\epsilon \geq M$ and $-k\epsilon \leq M$. Assume $k$ to be the smallest such integer. 

\begin{align*}
A_1 &= f^{-1}([0,\epsilon)) ,\\
A_2 &= f^{-1}([\epsilon, 2 \epsilon)),\\
A_3 &= f^{-1}([2\epsilon, 3\epsilon)),\\
&\vdots\\
A_k &= f^{-1}([(k-1)\epsilon, k\epsilon))
\end{align*}

\begin{align*}
A_{-1} &= f^{-1}([-\epsilon,0)),\\
A_{-2} &= f^{-1}([-2\epsilon, -\epsilon))\\
&\vdots \\
A_{-k} &= f^{-1}([-k\epsilon, -(k-1)\epsilon))
\end{align*}

The $A_i$ are disjoint and for $ I:= \{-k,...,-1,1,...,k\}$, $\cupdot_{i \in I} A_n\supset [-M,M]$. So for each $x \in [a,b]$ such that $|f(x)| \leq M$, $f(x)$ lies within exactly one of the $A_i$. Let $\alpha_i = i\epsilon - \epsilon/2$ for $i = 1,...,k$ and $\alpha_i = -i\epsilon + \epsilon/2$ for $i = -1,...,-k$. That is, $\alpha_i$ is the midpoint of the half open interval used in the definition of the set $A_i$. Define $\phi(x) = \sum_{i \in I} \alpha_i \chi_{A_i}(x)$ for each $x \in [a,b]$ such that $|f(x)| \leq M$. Then since each of the $A_i$ are measurable (this follows from the fact that $f$ is measurable), $\chi_{A_i}$ is measurable for each $i$ and so $\phi$ is measurable. Since $\phi$ is measurable and assumes only finitely many values, $\phi$ is a simple function. Let $x \in [a,b]$ such that $|f(x)| \leq M$. Then $x \in A_i$ for some $i \in I$ so $|f(x) - \phi(x)| = |f(x) - \alpha_i| \leq \epsilon/2 < \epsilon$. \\

If $m \leq f \leq M$ use a similar approach. Find $k \in \mathbb{N}$ such that $m + k\epsilon \leq M$ but for which $m + (k+1)\epsilon > M$. 

\begin{align*}
A_1 &= f^{-1}([m, m+\epsilon)) \\
A_2 &= f^{-1}([m + \epsilon, m+ 2\epsilon))\\
&\vdots \\
A_k &= f^{-1}([m + (k-1)\epsilon, m + k\epsilon))\\
A_{k+1} &= f^{-1}([m+k\epsilon, M]) \quad (\epsilon \geq M-(m+k\epsilon) \geq 0)
\end{align*}

Define $\alpha_1 = m + \epsilon/2$, $\alpha_2 = m + (3/2)\epsilon$,...,$\alpha_k = m+ (k-1/2)\epsilon$, $\alpha_{k+1} = (m+k\epsilon + M)/2$. That is, take $\alpha_i$ to be the midpoint of the interval used to define $A_i$.  Let $\phi = \sum_{i=1}^{k+1} \alpha_i\chi_{A_i}$. Then $\phi$ is a simple function. For each $x \in [a,b]$, $x \in A_i$ for exactly one of the disjoint $A_i$ and so $|\phi(x) - f(x)| < \epsilon$ and $m \leq \phi \leq M$. \\


(c) Given a simple function $\phi$ defined on $[a,b]$, there is a step function $g$ defined on $[a,b]$ such that $g(x) = \phi(x)$ except on a set of measure $\epsilon / 3$. If $m\leq \phi(x) \leq M$, then we may take $g$ so that $m \leq g \leq M$.  \\

Let $\phi : [a,b] \rightarrow \{\alpha_1, ... ,\alpha_n\}$, $\phi(x) = \sum_{i=1}^n \alpha_i \chi_{A_i}(x)$. Each $A_i$ is measurable and $m^*(A_i) = m(A_i) \leq m([a,b]) < \infty$. By Proposition 15 (ii) there is an open set $O_i'$ such that $m^*(O_i' \backslash A_i) < \epsilon / (6n)$. Let $O_i = O_i' \cap [a,b]$ for which we still have $A_i \subset O_i$ (since $A_i \subset [a,b]$ and $A_i \subset O_i'$) and $m^*(O_i \backslash A_i) < \epsilon / (6n)$. The open set $O_i$ can be written as a countable union of disjoint open intervals, $O_i = \cupdot_{n \in \mathbb{N}} I_n$.

\begin{align*}
m^*(O_i) &= m^*\left(\bigcupdot I_n\right) = m\left(\bigcupdot I_n\right)  = \sum_{n=1}^\infty m(I_n)
= \text{lim}_{N\rightarrow \infty} \sum_{n=1}^N m(I_n)\\
\implies &\exists N \text{ s.t. }  \sum_{N+1}^\infty m(I_n) < \epsilon / (6n) \quad \text{(the terms of a convergent series tend to zero)}\;.\\
&\text{Define } U_i := \cupdot_{n=1}^N I_n \\
m^*(O_i \backslash U_i) &= m^*(O_i \cap U_i^c) = m^*(\cupdot_{n=N+1}^\infty I_n) = m(\cupdot_{n =N+1}^\infty I_n) = \sum_{n =N+1}^\infty m(I_n) < \epsilon / (6n)
\end{align*}

Repeat this process for each $A_i$ to produce a $U_i$. Since each $U_i$ is a union of open intervals, we can define the step function $g = \sum_{i = 1}^n \alpha_i \chi_{U_i}$. Since the $U_i$ are disjoint, for each $x \in [a,b]$, $x \in U_i$ for at most one $U_i$. Then $g(x) =\alpha_i = \phi(x)$ for $x \in A_i \cap U_i$. So for each $i$ we have $\phi(x) = \alpha_i = g(x)$ except on $U_i \triangle A_i$ and:
\begin{align*}
m(U_i \triangle A_i) &= m((U_i \backslash A_i) \cup (A_i \backslash U_i)) \\
&\leq m(U_i \backslash A_i) + m(A_i \backslash U_i)\\
&=  m^*(U_i \backslash A_i) + m^*(A_i \backslash U_i)\\
&\leq m^*(O_i\backslash A_i) + m^*(O_i \backslash U_i)\\
&< \epsilon / (6n) + \epsilon / (6n) = \epsilon / (3n) \;.
\end{align*} 

In total, $\phi(x) = g(x)$ except on a set of measure $n\epsilon / (3n) = \epsilon / 3$. \\

If $m \leq \phi(x) \leq M$, define $g$ just as before except that instead of using $\chi_{U_i}$, use

$$\chi_{U_i}' = \begin{cases}
1 & x \in U_i \\
m & x \not\in U_i
\end{cases} \;.
$$

Then whenever $\phi(x) = g(x)$, it must be that $m\leq g(x) \leq M$ and whenever $\phi(x) \neq g(x)$ we still have either $m\leq g(x) = \alpha_i \leq M$ for some $\alpha_i$ or $g(x) = m \leq M$. This change accounts for the possibility that $m > 0$ which would allow $g(x) = 0 < m$ to occur. \\

(d) Given a step function $g$ defined on $[a,b]$ there is a continuous function $h$ defined on $[a,b]$ such that $g(x) = h(x)$ except on a set of measure $\epsilon / 3$. If $m\leq g \leq M$ we can take $h$ such that $m\leq h \leq M$. \\

Let $g$ be a step function defined on $[a,b]$. Then $g$ can be written in the form $g = \sum_{i = 1}^m \alpha_i \chi_{I_i}$ where the $I_i$ are intervals. The intervals can be taken so that they are disjoint and $\cupdot_{i=1}^m I_i = [a,b]$. Also, if it is the case that $I_j$ and $I_k$ are consecutive intervals and $\alpha_j = \alpha_k$ then we can collapse $I_j$ and $I_k$ into a single interval $I_l = I_j \cup I_k$ on which $g(x) = \alpha_j = \alpha_k =: \alpha_l$. Relabelling if necessary, assume $a$ is the left endpoint of the interval $I_1$ and $b$ the right endpoint of the last interval, $I_m$, used in the definition of $g$. We will use a construction that does not depend on whether the endpoints of any particular interval are open or closed (to include $a$ and $b$ we need at least two closed endpoints). For purely notational convenience, therefore, we will write most of the intervals as if they were all open - but it should be understood that these intervals may not take this form. Write

\begin{align*}
I_1 &= [p_0, p_1) = [a,p_1) \\
I_2 &= (p_1, p_2) \\
&\vdots \\
I_n &= (p_{n-1}, p_n] = (p_{n-1}, b] \;.
\end{align*}

The function $g$ is discontinuous at the $n-1$ points $p_1, ..., p_{n-1}$. We define $h$ to be equal to $g$ except at intervals of the size $\epsilon / [3(n-1)]$ around each of these points of discontinuity (if $n = 1$, $g$ is constant on $[a,b]$ and so already continuous itself). Let $\Delta x = \epsilon / [3(n-1)]$. 

$$h(x) = \begin{cases}
\alpha_1 &  a\leq x \leq p_1 - \Delta x / 2 \\
\frac{\alpha_2 - \alpha_1}{\Delta x}[x - (p_1 - \Delta x / 2)] + \alpha_1 & p_1 - \Delta x / 2 < x < p_1 + \Delta x / 2 \\
\alpha_2 & p_1 + \Delta x / 2 \leq x \leq p_2 - \Delta x /2 \\
\frac{\alpha_3 - \alpha_2}{\Delta x}[x - (p_2 - \Delta x / 2)] + \alpha_2 & p_2 - \Delta x / 2 < x < p_2 + \Delta x / 2 \\
\vdots \\
\frac{\alpha_n - \alpha_{n-1}}{\Delta x}[x - (p_{n-1} - \Delta x / 2)] + \alpha_{n-1} & p_{n-1} - \Delta x / 2 < x < p_{n-1} + \Delta x / 2 \\
\alpha_n & p_{n-1} + \delta x / 2 \leq x \leq b
\end{cases}$$

The result is that $h$ is constant and agrees with $g$ except near the 'jumps' of $g$, where $h$ is then defined to be a linear function connecting each of the constant portions in order to satisfy continuity. Then $h$ disagrees with $g$ on $n-1$ intervals each of length $\epsilon / [3(n-1)]$. That is, $h$ is a continuous function for which $h(x) = g(x)$ except on a set of measure $\epsilon/3$. Using this construction of $h$, it follows that if $m\leq g \leq M$, then $m\leq h \leq M$ as well. \\

{\bf Conclusion} Let $f$ be a measurable function defined on $[a,b]$ and assume that $f$ takes on the values $\pm \infty$ only on a set of measure zero. Then given $\epsilon > 0$, there is an $M$ such that $|f| \leq M$ except on a set $A$ of measure less than $\epsilon / 3$ by (a). By (b) there is a simple function $\phi$ such that $|f - \phi| < \epsilon$ except where $|f| \geq M$. By (c) there is a step function $g$ such that $\phi = g$ except on a set $C$ of measure less than $\epsilon / 3$. So $|f - g| = |f-\phi| < \epsilon$ except possibly on $A\cup C$ where $m(A\cup C) < 2\epsilon / 3 < \epsilon$. By (d) there is a continuous function $h$ such that $g = h$ except on a set $D$ of measure less than $\epsilon / 3$. So $|f - h| = |f-g| = |f-\phi | < \epsilon$ except possibly on $A\cup C \cup D$ with $m(A\cup C \cup D) < 3 \epsilon / 3 = \epsilon$. The results are only improved if $m \leq f \leq M$ since in this case we can find $\phi$ such that $|f - \phi| < \epsilon$ over a more inclusive set. 

\subsection*{6 Littlewood's Three Principles}

{\bf Proposition 23} Let $E$ be a measurable set of finite measure, and $(f_n)$ a sequence of measurable functions defined on $E$. Let $f$ be a real-valued function such that $f_n \rightarrow f$ pointwise on $E$. Then given $\epsilon > 0$ and $\delta >0$, there is a measurable set $A \subset E$ with $m(A) < \delta$ and an integer $N$ such that for all $x \not\in A$ and for all $n \geq N$, 

$$|f_n(x) - f(x)| < \epsilon \;.$$

Proof: Let 
$$G_n = \{x \in E : |f_n(x) - f(x)| \geq \epsilon\}$$
$$E_N = \bigcup_{n=N}^\infty G_n = \{x \in E : |f_n(x) - f(x) | \geq \epsilon \text{ for some } n\geq N\} \;.$$

We have $E_{n+1} = \bigcup_{n=N+1}^\infty G_n \subset \bigcup_{n = N}^\infty G_n =  E_n$. Since for each $x \in E$, $f_n(x)$ converges to $f(x)$, there must be some $E_N$ to which $E_N$ does not belong. Thus $\cap_{N \in \mathbb{N}} E_n = \emptyset$, and so by Proposition 14 (note that $m(E_1)\leq m(E) < \infty$), $\text{lim}_{N\rightarrow\infty} \, m(E_N) = m\left(\cap_{N \in \mathbb{N}} E_N \right) = m(\emptyset) = 0$. Hence, given $\delta > 0$, there is an $N$ such that $m(E_n) < \delta$ for all $n\geq N$. In particular $m(E_N) < \delta$; that is,

$$m(E_N) = m(\{x \in E : |f_n(x) - f(x)| \geq \epsilon  \text{ for some } n \geq N\}) < \delta \;.$$

If we set $A = E_N$ for this $N$, then $A \subset E$ is measurable and $m(A) < \delta$ and $$A^c = \{x \in E : |f_n(x) - f(x)| < \epsilon \text{ for all } n \geq N\} \;.$$

To see this:  $x \in A^c \iff x \not\in A \iff x \not\in \{y \in E : |f_n(y) - f(y)| \geq \epsilon \text{ for some } n \geq N\}$. This is the case if and only if $|f_n(x) - f(x)| < \epsilon$ for all $n \geq N$, if and only if $x \in \{y \in E : |f_n(y) - f(y)| <\epsilon \text{ for all } n \geq N\}$. That is, as the proposition states, for all $x \not\in A$, for all $n\geq N$, $|f_n(x) - f(x)| < \epsilon$. \\

{\bf Proposition 24} Let $E$ be a measurable set of finite measure, and $(f_n)$ a sequence of measurable functions that converge to a real-valued function $f$ almost everywhere on $E$ (the set $B$ of points such that $f_n$ does not converge to $f$ pointwise is such that $m(B) = 0$ and $f_n \rightarrow f$ pointwise on $E \backslash B$). Then given $\epsilon > 0$ and $\delta > 0$, there is a set $A \subset E$ with $m(A) < \delta$, and an $N$ such that for all $x\not\in A$ and for all $n\geq N$, 

$$|f_n(x) - f(x)| < \epsilon \;.$$

Proof: By definition of $f_n$ converging to $f$ almost everywhere, the set $B$ is measurable and so $E\backslash B$ is measurable with finite measure and the $f_n$ are defined on $E \backslash B$ (by restriction if necessary). Given $\epsilon> 0$ and $\delta > 0$, then applying Proposition 23 to $E \backslash B$, there is a set $C\subset E$ with $m(C) < \delta$ and an integer $N$ such that for all $x \in (E\backslash B) \backslash A = E \backslash (B \cup C)$ and for all $n \geq N$, $|f_n(x) - f(x)| < \epsilon$. But this means for $A = B\cup C$, that $A\subset E$, $m(A) \leq m(B) + m(C) < \delta$, and for all $x \not\in A$ and for all $n\geq N$, $|f_n(x) - f(x)| < \epsilon$. \\

{\bf Problem 30} (Proving {\bf Egoroff's Theorem}) If $(f_n)$ is a sequence of measurable functions that converge to a real-valued function $f$ almost everywhere on a measurable set $E$ with $m(E) < \infty$, then given $\eta > 0$, there is a subset $A \subset E$ with $m(A) < \eta$ such that $f_n \rightarrow f$ uniformly on $E\backslash A$. \\

Proof:  Let $\eta > 0$. For every $n \in \mathbb{N}$, there exists by Proposition 24 a measurable set $A_n \subset E$ such that $m(A_n) < \delta_n:= 2^{-n}\eta$ and an $N_n$ such that for all $k \geq N_n$ and all $x \in E\backslash A_n$, $|f_k(x) - f(x)| < \epsilon_n := 1/n$. Since each $A_n\subset E$ is measurable, $\cup_{n=1}^\infty A_n$ is measurable and $\cup_{n=1}^\infty A_n \subset E$. Let $A:= \cup_{n=1}^\infty A_n$. By Proposition 13 (subadditivity), 

$$m(A) \leq \sum_{n=1}^\infty m(A_n) < \sum_{n=1}^\infty 2^{-n}\eta = \eta \;.$$

So $A$ is a subset of $E$ such that $m(A) < \eta$. If we can show that $f_n$ converges uniformly to $f$ on $E \backslash A$ then we will have proven Egoroff's Theorem. Now let $\epsilon > 0$ be given. Then there is an $m$ such that $A_m \subset E$ and a corresponding $N_m$ such that for all $k\geq N_m$ and $x \in E\backslash A_m$, $|f_k(x) - f(x)| < 1/m < \epsilon$. But then for $x \in E\backslash A$, $x \not\in \cup_{n=1}^\infty A_n$. In particular, $x \not\in A_m$. Therefore, given $\epsilon > 0$, there is a positive integer $N_m$ such that for all $k\geq N_m$ and $x \in E\backslash A$, $|f_k(x) - f(x)| < 1/m < \epsilon$. Conclude that there is a measurable set $A\subset E$ with $m(A) < \eta$ such that $f_k \rightarrow f$ uniformly on $E\backslash A$.\\

{\bf Problem 31} (Proving {\bf Lusin's Theorem}) Let $f$ be a measurable real-valued function on $[a,b]$. Given $\delta > 0$, there is a continuous function $\phi$ on $[a,b]$ such that $m(\{x : f(x) \neq \phi(x)\}) < \delta$. \\

Proof: Since $f$ is real-valued, $m(\{x : f(x) = \pm \infty\}) = m(\emptyset) = 0$. For each $k \in \mathbb{N}$, there exists by Proposition 22 a continuous function $h_k$ such that $|f - h_k| < 1/k$ except on a set of measure less than $1/k$, i.e. $m(\{x \in [a,b] : |f(x) - h_k(x)| \geq 1/k \}) < 1/k$. Then the sequence $(h_k)$ of continuous (and therefore measurable) functions converges to $f$ almost everywhere on $[a,b]$. By Egoroff's Theorem, there is a set $A \subset [a,b]$ with $m(A) < \delta / 2$ such that $h_k \rightarrow f$ uniformly on $[a,b] \backslash A$. By Proposition 15 (iii) (more specifically (i) $\iff$ (iii) and the fact that $[a,b] \backslash A$ is a measurable set), there exists a closed set $F \subset ([a,b] \backslash A)$ such that $m^*(([a,b] \backslash A) \backslash F) < \delta / 2$. Since $F \subset [a,b] \backslash A$, $h_k \rightarrow f$ uniformly on $F$. Since the convergence of $(h_k)$ to $f$ is uniform on $F$, $f$ must be continuous on $F$. By Problem 2.40, there exists a continuous function $\phi$ defined on $(-\infty, \infty)$ (and so $\phi$ is defined and continuous on $[a,b]$) such that $\phi(x) = f(x)$ for all $x \in F$. Therefore, if $x\in [a,b]$ and $f(x) \neq \phi(x)$, it must be the case that $x \not \in F$. So $x \in ([a,b] \backslash A) \backslash F$ or $x \in A$. That is,

$$m(\{x \in [a,b] : f(x) \neq \phi(x) \}) = m\left[\left(([a,b] \backslash A) \backslash F\right) \cup A\right] \leq m\left(([a,b] \backslash A) \backslash F\right) + m(A) < \delta/2 + \delta / 2 = \delta \;.$$
\end{document}